
\section{Review of State of the Art}
\subsection{Extreme Value Theory}
\subsubsection{Asymptotic Dependence}
A simple measure of asymptotic dependence between two variables sharing the same
  marginal distribution is the coefficient of asymptotic dependence, $\Chi$
\begin{equation}
    \Chi_{ij} = \lim_{z\rightarrow\infty}\text{Pr}(Z_i > z \mid Z_j > z)
\end{equation}
\cite{rootzen2006} defines the multivariate Generalized Pareto distribution,
which is expanded on in \cite{ferreira2014}, introducing the multivariate
Generalized Pareto process.\makenote{need to go through rootzen 2018}



\subsubsection{Generalized Pareto Process}






\subsection{Anomaly Detection}
Anomaly detection is a broadly defined term, but in this context we choose it to
  mean finding observations that are \emph{different} in some capacity from the
  rest of the observations in the data.  The lion's share of anomaly detection
  algorithms can be summarized into two main categories: distance based methods,
  and density based methods.

\subsection{Distance based methods}
Distance based methods work with some notion of distance to define how unique an
  observation is by how far it is from other observations in the data.  A simple
  illustrative example might be minimum distance to neighbor, for each
  observation.  The algorithm would proceed as follows: first scale the data,
  then calculate pairwise distances between each observation. Those observations
  with the highest minimum distance to a neighbor are the most unique, and
  therefore seem the most anomalous.  \makenote{Insert summaries and citations
  of distance methods} This basic idea is extended in several ways through
  clustering methods \makenote{insert citation of k-means}, decision
  trees\makenote{insert citation of isolation forests}, and so on.  A basic
  assumption required we can gather from this example is that we expect
  non-anomalous data to behave in a consistent manner, and anomalous data to
  behave in unique and different manners.  These methods generally make little
  to no assumptions regarding the underlying distribution of the data.

\makenote{isolation forests, DBSCAN, }

\subsection{Density based methods}
\makenote{complete rewrite of this paragraph.  it sucked.}

\makenote{Local outlier factor--both camps}

\subsubsection{Observation Density}
Under this approach, an illustrative example would be to look at the
  contribution to the marginal likelihood for each observation, and identify as
  anomalies those observations which are least likely.
\makenote{expand}

\makenote{something something posterior probability of observation,
          lowest contribution to log-likelihood, etc.}

The defining characteristic of this category is we model all the data, and don't
  model anomalous data as having come from a separate distribution.  Then, we
  look at how likely is that observation.  The observations that are least
  likely given the model are considered anomalous.  We might consider
  studentized residuals from linear regression as being something akin to this.

Our method falls somewhat into the density based method idea.  We are asserting
  a parametric model upon the data, but our method requires one additional
  assumption as to what it means to be anomalous: we assert that anomalies must
  be \emph{extreme} in at least one margin.

% EOF
