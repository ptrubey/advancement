
The field of extreme analysis provides a means of extrapolating information from observed data into
  predicting behavior of tails of a distribution.  One sub-field of extreme analysis focuses on modelling 
  tail behavior by exceedences over a threshold, using the Pareto distribution.  In recent years much 
  work has been done in exploring the definition and properties of an appropriate generalization of the
  univariate generalized Pareto distribution for threshold exceedences to a multivariate setting.  This 
  paper builds on the constructive definition of the multivariate Pareto presented in \cite{ferreira2014}
  that decomposes the vector of interest into independent radial and angular components; the latter 
  supported on a particular manifold and containing all and only information relevant to the dependence
  structure of the distribution.  We motivate our analysis with a discussion of extreme analysis applications
  in atmospheric sciences, in particular using the integrated vapor transport (IVT) data for assessing
  atmospheric rivers.  This data covers approximately 30 years, and provides daily measurements of atmospheric water in grid cells covering California.

In this advancement document, we propose a novel approach to parameterize this dependence structure of 
  the multivariate Pareto, and conduct inference upon it.  We discuss criteria by which we can evaluate our
  proposed models, as the manifold the dependence structure is supported on does not lend itself to using the
  Euclidean distance metric.  Using our motivating example, we then explore some opportunities afforded to 
  us by a parametrically  modelled dependence structure in classical multivariate extreme analysis.
  
We review available methods of anomaly detection, and then leverage our proposed model of the dependence
  structure to develop methods of anomaly detection.  We propose two such methods, and in preliminary 
  analysis on simulated data, we show them to be competitive with existing methods.
  
Finally we discuss computational developments that may allow us to apply our model at scale.  We discuss a
  possible motivating example in the form of storm surge, where a given vector may have hundreds of 
  thousands to millions of elements.  Application at this scale will require further development to
  maintain model fidelity in high dimensions.
 
% EOF
