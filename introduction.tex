
\section{Introduction}
Atmospheric rivers are temporary events, where large elongated regions of high concentrations of
  water vapor are developed in the atmosphere and carry huge amounts of water potentially thousands
  of miles.  The amount of water in transit during these events dwarfs that of terrestrial rivers.
  For the targeted region, the atmospheric river can represent a significant portion of the
  precipitation the region will experience.  Such events are thus of great interest to meteorologists,
  as well as farmers, \makenote{expand and cite}.

One metric by which we might identify and declare atmospheric rivers is the integrated water vapor
  transport, or \emph{IVT} \bruno{Why does `` emph '' produce an underlying word instead of italics?
  It messes up the style of the references}\makenote{I don't know.  It doesn't when I build it on my
  computer.  This is only happening in overleaf...}.  This value represents the total amount of water
  vapor being transported in an atmospheric column--that is, a column of the troposphere of particular
  size. These values can be measured by dropsondes, but the values we are using are estimated as part
  of a data product\makenote{needs citation}.  An observation from these data includes a reading (estimated
  or measured) at grid cell at a period in time; where a grid cell represents the surface of the
  earth associated with the atmospheric column.  Our specific data comes from the coast of California,
  with daily readings of IVT covering 30 years, omitting leap days.

We have two such datasets, in differing spatial resolutions.  The lower resolution splits the coast
  of california into 8 grid cells, while the higher resolution does so into 46 grid cells.
  \bruno{
  At this point you have to say what is the problem with these data that is relevant for this work.
  I suppose the story could something like: we are interested is extreme values of IVT, and in how
  likely it is that an extreme IVT will happen jointly in more than one location, as well as in
  identifying configurations of the IVT values at the different location that may be anomalous.
  The you say that you will propose different models and that you will evaluate them using
  the two different datasets.
  }
  We will be looking at relative model performance on these two datasets as a means of evaluating
  how well any model we propose scales.  We are specifically interested in extremal dependence--the
  relationship between the upper tails of dimensions of the distribution.  In this case, that means
  the relationship between extreme values in different grid cells.  We are going to be looking at
  point-in-time behavior rather than considering the time series nature of the data--that
  relationship may come later.

As we are interested in the extremal dependence, it makes sense that we would choose to represent
  this data using tools from extreme value theory.  Extreme value theory, or \emph{EVT}, seeks to
  model and assess probability of observing extreme events.  Such a topic is applicable generally,
  but it finds particularly strong use among such fields as finance \citep{allen2013},
  climatology \citep{trepanier2018}, and insurance \citep{beirlant1994}.  In these fields, extreme
  events may represent significant loss to the body commissioning the study.  For instance, an
  insurance company might commission a study on extreme weather events, as an extreme weather event
  localized to a particular region could cause a spike in claims from that region.  Extreme value
  theory offers us a tool set for making inference about the tails of a distribution, without having
  observed said tails.  For
  instance, with an extreme weather event like flooding, we can make predictions about return
  levels--the average time until an observation of a particular magnitude occurs--without having seen
  an observation of that magnitude, or having observed that long.  In the context of our motivating
  example, extreme values in the IVT may represent the formation of an atmospheric river, which has
  dramatic effects on precipitation and ground water.  The formation of an IVT may lead to flooding
  and other negative effects, but is also contributes necessary water for irrigation and agriculture.
  Development of statistical tools regarding these extreme events becomes necessary for making informed
  decisions.

\subsection{Univariate EVT - Maxima}
Regarding the asymptotic behavior of extreme events, there are a couple major strategies for conducting
  inference--developing probabilistic estimates of extreme behavior.  First developed is the theory
  of a limiting distribution on maxima--the largest observation from a sample.  For a sample ${\bf x}$
  where ${\bf x} = (x_1,\ldots,x_n)$ represents a sequence of $n$ independent random variables from a
  distribution function $F$, the distribution of the maximum $M_n$ of this sequence can be derived
  as:
  \begin{equation*}
    \begin{aligned}
      \text{Pr}(M_n\leq z) &= \text{Pr}(X_1 \leq z, \ldots, X_n \leq z)\\
        &= \text{Pr}(X_1\leq z)\times\ldots\times\text{Pr}(X_n\leq z)\\
        &= F(z)^n.
    \end{aligned}
  \end{equation*}
  In situations where $F$ is unknown, we can seek to approximate the behavior of $F^n$ as
  $n\rightarrow\infty$.  To ensure this does not degenerate to a point mass, we need to consider a
  standardized sequence of maxima. If this sequence stabilizes as $n$ increases, the a limiting
  distribution exists.  More specifically, if there exists some sequence of constants $a_n > 0$, $b_n$
  such that:
  \begin{equation*}
    \text{Pr}\left[\frac{M_n - b_n}{a_n} \leq z\right] \stackrel{d}{\rightarrow} G(z)
  \end{equation*}
  as $n\rightarrow\infty$, then $G$ is a max-stable distribution, and $F$ is in the domain of
  attraction of that max stable distribution.  Maurice Fr{\'e}chet \cite{frechet1927} originates the
  field, identifying two limiting forms of max-stable distributions, which would become known as the
  Fr{\'e}chet and Weibull distributions. \cite{weibull1951} expands the analysis of the Weibull
  distribution.  \cite{fisher1928}, in what is now called the \emph{Fisher-Tippett Theorem}, identifies
  the Fr{\'e}chet and Weibull distributions, along with an as then unnamed third form, as the three
  limiting forms of the distribution of the maxima of a sample.  \cite{gumbel1935,gumbel1942} offers
   an analysis of that third form, now known as the Gumbel distribution.  Later works, including
   \cite{jenkinson1955} reparameterize all three forms as special cases of a single unifying form,
   the generalized extreme value distribution, \emph{GEV}:
  \begin{equation*}
    \label{eqn:gev}
    F(m \mid \mu, \sigma, \xi) = \exp\left\lbrace-\left[1 + \xi\left(\frac{x - \mu}{\sigma}\right)\right]_{}^{-1/{\xi}}\right\rbrace.
  \end{equation*}
  Thus distributions in the domain of attraction of an EVD (Gumbel, Fr{\'e}chet, or Weibull) will be
  in the domain of attraction of the GEV.

As this distribution specifies asymptotic behavior for the maximum of a set of observations,
  inference assuming this distribution requires we specify some block of data that we would take the
  maximum in, and for the block report only that maximum.  A series of these blocks yielding a series
  of maxima allows us to conduct inference about the parameters of the distribution.  Taking only the
  maximum in a block of observations necessarily leads to the reduction of our sample size by a factor
  of $1/\text{block size}$. In problems where the data occur in natural blocks, such as an hourly time
  series where a natural block might be a day, this might be appropriate.  There is an implicit
  violation of the assumption of independence within a block, in most cases, but that violation is
  generally ignored. In data without a natural block, it may be difficult to justify an induced
  artitifial block.\makenote{need more better sentence.} Moreover, in general, retaining only one
  datum for each block increases the variability of the parameter estimates.

\subsection{Univariate EVT - Peak Over Threshold}
The \emph{Pickands-Balkema-de Haan Theorem}\citep{balkema1974,pickands1975} offers us another means
  of conducting inference that will be less wasteful of data.  If the original distribution $F$ is
  in the domain of attraction of the GEV, then excesses over a high threshold $u$ can be well modelled
  using the Generalized Pareto distribution.  Again, let $X$ follow some distribution function $F$.
  It follows that:
  \begin{equation*}
    \text{Pr}\left[X > u + y\mid X > u\right] = \frac{1 - F(u + y)}{1 - F(u)}
  \end{equation*}
  for $y > 0$.  If $F$ is in the domain of attraction of an EVD, then
  \begin{equation*}
    \lim\limits_{u\to u^{\prime}}\text{Pr}\left[X > u + y\mid X > u\right] = H(y)
  \end{equation*}
  where $u^{\prime}$ represents some upper boundary of $X$ has a functional form--the survival
  function of a Pareto distribution. We approximate this asymptotic relationship by setting some
  large threshold $u$, and then threshold exceedences $Y = X - u$ are modelled under the distrribution
  function
  \begin{equation*}
    \label{eqn:gp}
    H(y) = 1 - \left(1 + \xi\frac{y}{\sigma}\right)^{-\frac{1}{\xi}}.
  \end{equation*}
  This defines the generalized Pareto family of distributions.  Thus, if block maxima have a
  limiting distribution $G$ within the EVD family, then threshold exceedances for a sufficiently
  high threshold have a limiting distribution $H$ within the Generalized Pareto (GP) family.  We can
  identify $\sigma$ as a scale parameter, but $\xi$ deserves special mention as the extremal index.
  We can interpret the Pareto tail probability as
  \begin{equation*}
    \lim\limits_{t\to\infty}\frac{1 - F(ty)}{1 - F(t)} = y^{-\frac{1}{\xi}}
  \end{equation*}
  for $y > 1$.  Importantly, $\xi$ from the limiting distribution of excesses over a threshold is the
  same parameter, and has the same value as the $\xi$ from the limiting distribution of the maximum
  observation of a sample.  Pickands\needcite and Hill\needcite provide estimators for this value
  relying on this result.

One other point of note is that for time series, the implicit assumption of independence for those
  observations in excess of a threshold is violated.  One means of dealing with this violation is
  to consider a string of observations in excess of a threshold as correlated, and only keep one
  of the string.\makenote{Need citation of paper that uses this approach...as we do as well.}

\subsection{Multivariate EVT}
The generalization of EVT to multiple dimensions presents a unique challenge in the case of maxima--how
  do we define the maximum of a sample?  How can we order observations in a multivariate scenario?
  \cite{rootzen2006} offers a brief review of the subject.  One possible treatment sidesteps this
  conundrum by using component maxima--let ${\bf M}_n = \left(\vee_i X_{i1},\ldots,\vee_i X_{id}\right)$.
  Then the multivariate GEV establishes the limiting behavior of ${\bf M}_n$ as $n\to\infty$.  Note
  that ${\bf M}_n$ need not be in the observed $\left\lbrace{\bf X}_i; i = 1,\ldots,n\right\rbrace$.
  Then, if there is a set of constants ${\bf a}_n > {\bf 0}$, ${\bf b}_n$ such that
  \begin{equation*}
    \lim\limits_{n\to\infty}\text{P}\left(\cap_{l = 1}^d \frac{M_{nl} - b_{nl}}{a_{nl}} \leq x_l\right) = G({\bf x})
  \end{equation*}
  with non-degenerate marginals, then $G({\bf x})$ is a \emph{multivariate extreme value distribution}.
  The marginal distribution of $M_{nl}$ is by construction a GEV with the parameters $a_{nl}, b_{nl}, \xi$.
  With the existence of the marginal distributions established as a consequence of the existence of the
  GEV, multivariate EVT then splits into two components--estimation of the marginal parameters, and
  estimation of the dependence structure.  This separation is born out extensively in the literature.

\makenote{Need to add in further work on multivariate Pareto origin; exponent measure}

In the same manner we extended the single variable peaks over threshold approach to the GEV, we can
  extend a multivariate peaks over threshold approach to the multivariate GEV.  \cite{rootzen2006}
  defines the multivariate generalized Pareto from a theoretical starting point in the GEV.  Indeed,
  they define of the generalized Pareto in the form:
  \begin{equation*}
    H({\bf x}) = \frac{1}{-\log G({\bf 0})}\log\frac{G({\bf x}}{G({\bf x}\wedge {\bf 0})},
  \end{equation*}
  where $G$ is a GEV as previously defined.  Further analysis is provided in \cite{falk2008}.
  \cite{rootzen2018} provides a modern review of state of the art multivariate peaks over thresholds
  modelling using the generalized Pareto.

As in the multivariate extrema case, analysis in peaks over thresholds modelling generally first
  develops the marginal distributions, then establishes the dependence structure.  In extreme analysis,
  a frequently used method for describing the dependence structure is the copula\needcite, but we
  propose a novel method based on the constructive definition of the multivariate Pareto presented in
  \cite{ferreira2014}.  We follow the approach used in Section~2.2 of \cite{goix2015}.  Assume that
  the $d$-dimensional vector ${\bf Z}$, after standardization is such that for some distribution
  $\mu$ defined on the positive quadrant of $\mathcal{R}^d$,
  \begin{equation}
    \lim\limits_{n\to\infty}n\text{Pr}\left(\frac{1}{n}{\bf Z}\geq {\bf z}\right) = \mu\left([{\bf 0},{\bf z}]^c\right).
  \end{equation}
  In this setting, $\mu$ is the asymptotic distribution of ${\bf Z}$ in extreme regions--the
  \emph{exponent measure}.  $\mu$ features the homogeneity property $\mu(tA) = \frac{1}{t}\mu(A)$.
  By this property, \cite{ferreira2014} factorize ${\bf Z}\to R{\bf V}$, where
  $R = \pnorm{{\bf Z}}{\infty} > 1$ describes the radial component, and
  ${\bf V} = {\bf Z}/\pnorm{{\bf Z}}{\infty} \in \mathcal{S}_{\infty}^{d-1}$ the angular component of
  ${\bf Z}$ in $\mathcal{R}_+^d$.  $\mu(\cdot)$ is similarly factorized as
  \begin{equation}
    \mu\left( [{\bf Z} : R({\bf Z}) > r, V({\bf Z}) \in A ] \right) = r^{-1}\Phi(A),
  \end{equation}
  where $\Phi(\cdot)$ is the \emph{spectral measure}.  The spectral measure is thus independent and
  separable from the radial component.  Furthermore,
  \begin{equation}
    \text{P}\left({\bf V} \in A \mid R > r\right)
      = \frac{r\text{P}\left({\bf V} \in A, R > r\right)}{r\text{P}(R > r)}
      \rightarrow_{r\to\infty} \frac{\Phi(A)}{\Phi(\mathcal{S}_{\infty}^{d-1})}
  \end{equation}
  Describing the dependence structure of ${\bf Z}$, equivalently describing the spectral measure
  $\Phi$.  Much literature exists describing this dependence structure using copulas\needcite.
  We seek instead to establish a distribution on ${\bf V} \in \mathcal{S}_{\infty}^{d-1}$.

To accomplish this task, first we must standardize ${\bf X}$ to a common marginal distribution.
  Following the work of \cite{ferreira2014}, if the marginal distributions $F_{l}$ fall into the domain
  of attraction of some GEV $G_l$, then threshold exceedences for a sufficiently high threshold can
  be modelled as generalized Pareto $H_l$.  Standardization occurs as
  \begin{equation}
    Z_l = \left(1 + \xi_l\frac{X_l - b_{t,l}}{a_{t,l}}\right)_{+}^{1/\xi_l}.
  \end{equation}
  We establish the marginal thresholds $b_{t,l} := \hat{F}_l^{-1}\left(1 - \frac{1}{l}\right)$,
  then fit the scale parameters $a_{t,l}$, and extremal index $\xi_l$ via maximum likelihood.
  Note that $z_l > 1$ implies that $x_l > b_{t,l}$, meaning that the observation ${\bf x}$ is
  extreme in the $l$'th dimension.  Note also that $\sup_j Z_j$ follows a simple Pareto distribution.
  Fitting of the marginal extremal index and scale parameters uses only observations in excess of the
  marginal threshold. Following standardization, condition further analysis on $R \geq 1$.


%
% As the angular component is now independent of the radial component, we can establish a distribution
%   on solely the angular component.  For  $B\subset S_{\infty}^{d-1}$, We define the spectral
%   (or angular) measure, $\Omega(B)$, as
%   \begin{equation}
%     \Omega(B) = \mu[{\bf z}: R({\bf z}) > 1, {\bf V} \in B].
%   \end{equation}
%   \bruno{This needs t be sharpened. I don't see why $R({\bf z})>1$ is needed. Also
%   \[	\mu([0,1/\bm{z}]^c) = \int_{{\mathbb S}_\infty^{d-1}}
%   \left(\bigvee_{j} \theta_jz_j\right) d\Omega(\theta)\; ,\]
%   which is the reason $\Omega$ is a spectral measure.}
%   Then we can think of the spectral measure in terms of the limit measure $\mu$, and:
%   \begin{equation}
%     \mu[{\bf z}:R(z)>t, {\bf V}\in B] = t^{-1}\Omega(B).
%   \end{equation}
%   Thus we see a one-to-one correspondance between the limit measure $\mu$ and the
%     spectral measure $\phi$, and by factoring out the Pareto distributed radial
%     component, we can establish a distribution on the angular component
%   \begin{equation}
%     \text{Pr}\left({\bf V} \in B \mid r > 1\right) = \frac{\Omega(B)}{\Omega(S_{\infty}^{d-1})},
%   \end{equation}
%   \bruno{Again, this needs sharpening, I don't get the $r>1$. What you have is that, for any value of $r$.
%   \begin{equation}
%    \lim_{r\rightarrow\infty} \text{Pr}\left({\bf V} \in B \mid R > r\right) =
%               \frac{\Omega(B)}{\Omega(S_{\infty}^{d-1})},
%   \end{equation}
%   So, the point here is what we do in practice. We start by doing the univariate analysis for each
%   component. Take $b_{t,j} = F^{-1}(1 - 1/t)$ as the threshold. So
%   \begin{equation}
%    \lim_{t\rightarrow\infty} \text{Pr}\left({\bf V} \in B \mid R_t > 1\right) =
%             \frac{\Omega(B)}{\Omega(S_{\infty}^{d-1})},
%   \end{equation}.
%   What this means in practice is that we take high thresholds in each dimension, and then take
%       the standardized vectors for which $R>1$.}
%   \makenote{Paraphrasing from Goix et. al., figure if/how to cite}
%   conditioned on at least one of the components being extreme in the marginal
%   sense.  It is using this property that we will establish our method.
%
% For each $\nu\subset\lbrace{1,\ldots,d}, \nu \neq \emptyset$, we define the
%     \emph{truncated cone} $\mathcal{C}_{\nu}$, where
%   \begin{equation}
%       \mathcal{C}_{\nu} = \left\lbrace {\bf z} \geq 0 : \lVert z\rVert_{\infty}\geq 1, z_j \geq 0
%            \forall j \in \nu, z_j = 0 \forall j \not\in \nu\right\rbrace.
%   \end{equation}
%   That is, $\nu$ identifies a set index specifying components of the standardized
%   data for which the observation is greater than 0.  The observation is greater
%   than 0 for columns within that index, and 0 outside that index.  By
%   construction, we're also requiring that the observation is greater than 1 in
%   at least one of those dimensions.  By this definition, we observe that each
%   $\mathcal{C}_{\nu}$ is distinct and disjoint from any other
%   $\mathcal{C}_{\nu}$.  Now, defining $\Omega_{\nu}$ as the projection of
%   $C_{\nu}$ onto $S_{\infty}^{d-1}$,
%   \begin{equation}
%       \Omega_{\nu} = \left\lbrace {\bf v} \in S_{\infty}^{d-1} :
%                x_i > 0 \forall i \in \nu, x_i = 0 \forall i \not\in \nu\right\rbrace,
%   \end{equation}
%   we can clearly see $\mu(\mathcal{C}_\nu) = \Phi(\Omega_{\nu})$ for all
%   $\alpha \subset\lbrace1,\ldots,d\rbrace$.  Where we call $\mu(\dot)$ the limit
%   measure, we refer to $\Phi(\dot)$ as the \emph{spectral} or
%   \emph{angular measure}.  Our goal is establishing statistical inference on this
%   angular measure.




% EOF
