
\section{Introduction}
In recent years much work has been done in the exploration of the definition and properties of an
  appropriate generalization of the \emph{univariate} generalized Pareto distribution for exceedences
  over a threshold, to a \emph{multivariate} setting.  For a short accounting of the topic, the work of
  \cite{rootzen2006} defines the generalized Pareto distribution, with further analysis on these classes
  of distributions presented in \cite{falk2008} and \cite{michel2008}.  A recent review of the state
  of the art in multivariate peaks over threshold modelling using generalized Pareto is provided in
  \cite{rootzen2018}.  \cite{ferreira2014} presents a constructive definition of the multivariate Pareto
  that decomposes the random vector into independent radial and angular components; the latter
  containing all and only information pertaining to the dependence structure of the random vector.
  Based on this definition, we present a novel approach for modelling that angular component, and thus
  the modelling the dependence structure of the random vector.

To motivate the topic and our proposed solution, we acknowledge the relevance of extreme analysis to
  climatological events.  Outcomes such as heat waves, extreme precipitation, storm surges, and
  such present a similar challenge to investigators, in that the events of interest can be very destructive,
  and occur in probability at the tail end of \emph{normal} behavior.  In this fashion, a model built
  to capture \emph{normal} behavior may struggle to adequately represent extreme behavior.  More
  specialized tools become necessary; especially so in a time of intensifying extreme weather
  events~\citep{jentsch2007,vousdoukas2018,li2019}.  To this end, we use the integrated water vapor
  transport (\emph{IVT}) data for observing atmospheric rivers as an example application for our proposed method.

The first topic of the dissertation will concern the development of a model for multivariate extreme analysis,
  choosing to represent the dependence structure of multivariate Pareto using the projection of independent
  Gamma random variables onto the support of the angular component.  Analysis of this distribution follows.
  The second topic provides a review of the field of anomaly detection, and concerns contributions
  to the intersection of anomaly detection and extreme analysis that can be made using our model.
  The final topic discusses computational developments, and the application of extreme
  analysis--particularly our model--at scale.

The organization of the advancement document proceeds as follows:
  Section~\ref{sec:background} explains relevant background information; beginning with an overview of
  IVT, as well as a brief summary of field of extreme value theory.  Here we explain the specific space
  that the angular component of the multivariate Pareto is supported on. In Section~\ref{sec:methodology},
  we explore the difficulties in establishing a distribution on that support.  We introduce a family of
  manifolds that converge to our target space, and introduce a family of distributions supported on those
  manifolds.  In Section~\ref{sec:evaluation} we discuss what criteria are available to evaluate model
  fidelity, for the models we propose.  The nature of the space on which we build our model renders Euclidean
  distance inappropriate for model evaluation, so we discuss by what method we establish its replacement.
  Section~\ref{sec:results} presents metrics of model fidelity on simulated data for the proposed models,
  along with such metrics as evaluated on the IVT data.  In Section~\ref{ref:applications}, we discuss
  some of the possibilities afforded to us by such a parametric representation, and produce examples
  based on the IVT data, including pairwise extremal dependence coefficients, along with conditional
  survival curves using the posterior predictive distribution generated from our best model.
  Section~\ref{sec:anomaly} introduces the topic of anomaly detection, and provides a brief background
  of the field.  We also offer some methods of anomaly detection suited to multivariate EVT, and
  particularly suited to the statistical model we proposed. A preliminary analysis of our anomaly
  detection methods versus selected canonical methods is completed on simulated data, which shows
  our methods to be competitive.  In Section~\ref{sec:scale}, we discuss by what means we can scale
  this model to a large number of dimensions, along with a large number of observations.  Finally, we
  conclude, and present a timeline for the remaining work for the dissertation.

  % EOF
