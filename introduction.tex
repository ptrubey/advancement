
\section{Introduction}

 The integrated water vapor transport (IVT) is a metric describing the total amount of water vapor
  being transported in an atmospheric column.  The data is delivered as a vector, each element
  detailing the IVT for a particular area at a point in time.  Thus, we can consider the data to
  represent a multivariate time series.  The particular IVT data product we are working with
  \makenote{All of this needs to be rewritten... just getting ideas on paper}
  represents the coast of California, arranged in a set of grid cells.  One observation from this
  data set consists of readings from each grid cell at a point in time.  We have daily observations
  over 30 years, omitting February 29 in each leap year.

We have two IVT datasets, in differing spatial resolutions.  The lower resolution splits the coast
  of california into 8 grid cells, while the higher resolution splits California into 46 grid cells.
  We will be looking at relative model performance on these two datasets as a means of evaluating
  how well any model we propose scales.

We are interested in developing an understanding of extremal dependence between grid cells.  That is,
  if one grid cell's value is \emph{extreme}, how does that affect the distribution of other grid
  cells.  To that end, we employ some ideas from multivariate extreme value theory to recast our
  problem of extremal dependence as one of direction.
  
% EOF
