
\section{Introduction}
Atmospheric rivers are temporary events, where large elongated regions of high concentrations of
  water vapor are developed in the atmosphere and carry huge amounts of water potentially thousands
  of miles.  The amount of water in transit during these events dwarfs that of terrestrial rivers.
  For the targetted region, the atmospheric river can represent a significant portion of the
  precipitation the region will experience.  Such events are thus of great interest to meterologists,
  as well as farmers, \makenote{expand and cite}.

One metric by which we might identify and declare atmospheric rivers is the integrated water vapor
  transport, or \emph{IVT} \bruno{Why does `` emph '' produce an underlying word instead of italics? It messes up the style of the references}.  This value represents the total amount of water vapor being transported
  in an atmospheric column--that is, a column of the troposphere of particular size. These values
  can be measured by dropsondes, but the values we are using are estimated as part of a data
  product\makenote{needs citation}.  An observation from these data includes a reading (estimated
  or measured) at grid cell at a period in time; where a grid cell represents the surface of the
  earth associated with the atmospheric column.  Our specific data comes from the coast of California,
  with daily readings of IVT covering 30 years, omitting leap days.

We have two such datasets, in differing spatial resolutions.  The lower resolution splits the coast
  of california into 8 grid cells, while the higher resolution does so into 46 grid cells. \bruno{At this point you have to say what is the problem with these data that is relevant for this work. I suppose the story could something like: we are interested is extreme values of IVT, and in how likely it is that an extreme IVT will happen jointly in more than one location, as well as in identifying configurations of the IVT values at the different location that may be anomalous. The you say that you will propose different models and that you will evaluate them using the two different datasets.}
  We will be looking at relative model performance on these two datasets as a means of evaluating
  how well any model we propose scales.  We are specifically interested in extremal dependence--the
  relationship between the upper tails of dimensions of the distribution.  In this case, that means
  the relationship between extreme values in different grid cells.  We are going to be looking at
  point-in-time behavior rather than considering the time series nature of the data--that
  relationship may come later.

As we are interested in the extremal dependence, it makes sense that we would choose to represent
  this data using \st{lessons} \bruno{tools} from extreme value theory.  Extreme value theory, or \emph{EVT}, seeks
  to model and assess probability of observing extreme events.  Such a topic is applicable
  generally, but it finds particularly strong use among such fields as finance \citep{allen2013},
  climatology \citep{trepanier2018}, and insurance.\findcite  In these fields, extreme events may
  represent significant loss to the body commissioning the study.  For instance, an insurance
  company might commission a study on extreme weather events, as a localized extreme event could
  cause a spike in claims from that region.  Extreme value theory offers us a tool set for making
  inference about the tails of a distribution, without having observed said tails.  For instance,
  with an extreme weather event like flooding, we can make predictions about return levels--the
  average time until an observation of a particular magnitude occurs--without having seen an
  observation of that magnitude, or having observed that long.
  \bruno{This is good. Now you have to add a sentence illustrating the applicability for the IVT-atmospheric river story.}


\subsection{Univariate EVT - Maxima}
Extreme value theory describes the asymptotic behavior of extreme events. \bruno{You said that already. Now we need some glue to what you said in the previous paragraph.}  For a sample ${\bf x}$
  where ${\bf x} = (x_1,\ldots,x_n)$ represents a sequence of independent random variables from a
  distribution function $F$, the distribution of the maximum $M_n$ of this sequence can be derived
  as:
  \begin{equation*}
    \begin{aligned}
      \text{Pr}(M_n\leq z) &= \text{Pr}(X_1 \leq z, \ldots, X_n \leq z)\\
        &= \text{Pr}(X_1\leq z)\times\ldots\times\text{Pr}(X_n\leq z)\\
        &= F(z)^n
    \end{aligned}
  \end{equation*}
  Where $F$ is unknown, we seek to approximate the behavior of $F^n$ as $n\rightarrow\infty$.  To
  ensure this \st{doesn't} \bruno{does not} degenerate to a point mass, we \st{select a sequence of constants $a_n > 0$,
  $b_n$ and define $M_n^{\prime}$ as $M_n^{\prime} = (M_n - b_n)/a_n$, where $b_n$ represents the
  location, and $a_n$ the scale} \bruno{need to consider a standardized sequence of maxima. If this sequence stabilizes as $n$ increases, the a limiting distribution exists.} \st{These sequences stabilize as $n$ increases, which creates a
  limiting distribution for $M_n^{\prime}$. To summarize}, \bruno{More specifically,} if there exists some sequence of
  constants $a_n > 0$, $b_n$ such that:
  \begin{equation*}
    \text{Pr}\left[\frac{M_n - b_n}{a_n} \leq z\right] \stackrel{d}{\rightarrow} G(z)
  \end{equation*}
  as $n\rightarrow\infty$, then $G$ is a max-stable distribution, and $F$ is in the domain of
  attraction of that max stable distribution.  Maurice Fr{\'e}chet \citep{frechet1927} originates the
  field, identifying what would become known as the Fr{\'e}chet and Weibull distributions.  
  \cite{weibull1951} expands the analysis of the Weibull distribution; the results
  of that work giving it its current name.  \cite{fisher1928} identify the Fr{\'e}chet
  and Weibull distributions, along with a third form, as the three limiting forms of the distribution
  of the maxima of a sample.  \cite{gumbel1935} offered an analysis of that third form,
  what is now known as the Gumbel distribution.  Later works, including \st{that of}  \cite{jenkinson1955} reparameterize all three forms as special cases of a single unifying
  form, the generalized extreme value distribution, \emph{GEV}:
  \begin{equation*}
    \label{eqn:gev}
    F(m \mid \mu, \sigma, \xi) = \exp\left\lbrace-\left[1 + \xi\left(\frac{x - \mu}{\sigma}\right)\right]^{-1/{\xi}}\right\rbrace.
  \end{equation*}
  As we specified earlier, distributions can be in the domain of attraction of one of the
  aforementioned extreme value distributions.  If they are, regardless of which one, they will be in
  the domain of attraction of the GEV.  Now, one characteristic aspect of max stable distributions
  is that they feature the homogeneity property,
  \begin{equation*}
    \mu(tx) = \frac{1}{t}\mu(x).
  \end{equation*}
  Use of this property is \st{endemic in EVT} \bruno{fundamental in the generalization of EVT to a multivariate setting}.

As this distribution specifies asymptotic behavior for the maximum of a set of observations,
  inference assuming this distribution requires we specify some block of data that we would take the
  maximum in, and report only that maximum.  A series of these blocks yielding a series of maxima
  allows us to conduct inference about the parameters of the distribution.  Taking only the maximum
  in a block of observations \st{necessitates reducing} \bruno{leads to the reduction of} our sample size by a factor of $1/\text{block size}$.
  In \st{some} \bruno{problems where the} data \st{where there might} occur \st{a} natural block\bruno{s}, such as an hourly time series where a natural
  block might be a day, this might be appropriate.  There is an implicit violation of the assumption
  of independence within a block, \bruno{in most cases, }but that violation is generally ignored.  In data without a natural
  block, this data reduction might \st{be considered wasteful, as it limits our ability to conduct inference} \bruno{be artificial. Moreover, in general, retaining only one datum for each block increases the variability of the parameter estimates.
  }

\subsection{Univariate EVT - Thresholding}
Another way we can approach the problem that is less wasteful of data, is to specify a threshold,
  and consider \st{only those} observations \bruno{are} \emph{extreme} \st{that} \bruno{is they} are in excess of the threshold.  From the
  start, if $F$ is in the domain of attraction of an EVD, then for a random variable $X\sim F$,
  exceedances over a large threshold $u$ can be said to follow a Pareto distribution.  Again, let
  $X$ follow some distribution function $F$. Then let us regard observed values that exceed some
  threshold $u$ as extreme.  It follows that:
  \begin{equation*}
    \text{Pr}\left[X > u + y\mid X > u\right] = \frac{1 - F(u + y)}{1 - F(u)}
  \end{equation*}
  for $y > 0$.  If we consider the limit of the above as $u\to\infty$, then if $F$ is in the domain
  of attraction of an EVD, then $\lim\limits_{u\to\infty}\text{Pr}\left[X > u + y\mid X > u\right]$
  has a functional form--the survival function of a Pareto distribution. Let $X_1,\ldots,X_n$ be a
  sequence of random variables with the distribution function $F$.  Let $M_n = \max[x_1,\ldots,x_n]$.
  Suppose that $F$ is in the domain of attraction of the GEV, such that for large $n$,
  $\text{Pr}[M_n \leq z]\approx G(z)$.  Then, for large enough $u$, $\text{Pr}[X > u+y\mid x > u]$
  approximates to
  \begin{equation*}
    \label{eqn:gp}
    H(y) = 1 - \left(1 + \frac{\xi y}{\sigma}\right)^{-\frac{1}{\xi}}.
  \end{equation*}
  This defines the generalized Pareto family of distributions.  Thus, if block maxima have a
  limiting distribution $G$ within the EVD family, then threshold exceedances for a sufficiently
  high threshold have a limiting distribution $H$ within the Generalized Pareto (GP) family.
  Furthermore, the extremal index $\chi$ will be the same for these two limiting distributions.
  \bruno{Not clear where the extremal index comes from.}\makenote{I need to add some sources for
  Pareto distribution, as well as further explanation.  minor explanation of the existence for
  various estimators beyond ML for chi as well.}

One other point of note is that for time series, the implicit assumption of independence for those
  observations in excess of a threshold is violated.  One means of dealing with this violation is
  to consider a string of observations in excess of a threshold as correlated, and only keep one
  of the string.\makenote{Need citation of paper that uses this approach...as we do as well.}

\subsection{Multivariate Threshold EVT}
\makenote{Rewrite is done to this point.}
\bruno{At this point you need to say that in this work we will focus on the methods based on
  threshold exceedance in a multivariate setting. You should introduce the topic (although you might
  have already done that in the introduction) indicating the difficulties that there exist in the
  generalization of a Pareto distribution in the multivariate case. Cite some papers, of course.
  You can then say that the traditional approach consists on focusing on the estimation of the
  joint tail behavior, separate from the marginal behavior. }

Our motivating example lies in the domain of multivariate EVT, where we make inference about the
  distribution conditional on at least one threshold exceedance.  In multivariate EVT, the standard
  practice is to first standardize each dimension according to its marginal distribution, then
  estimate its dependence structure.\findcite  As has been stated, if each marginal distribution falls
  into the domain of attraction of the GEV, then threshold exceedances for a sufficiently
  \emph{sufficiently high} threshold will have a limiting distribution in the GP family\cite{beirlant2006}.
  Therefore, we estimate parameters for those marginal distributions considering only those
  observations that exceed the threshold.  For ${\bf x}$, for which the marginal distributions of its
  dimensions are within the domain of attraction of a GEV, standardization occurs as
  \begin{equation}
    Z_j = \left(1 + \xi_j\frac{X_j - b_{t,j}}{a_{t,j}}\right)_{+}^{1/\xi_j}.
  \end{equation}
  In practice, as estimation of these parameters is not the focus of this paper, we specify 
  $b_{t,j} := \hat{F}_j^{-1}\left(1 - \frac{1}{t}\right)$, then fit the scale parameter $a_{t,j}$,
  and extremal index $\xi_j$ via maximum likelihood.  Note that $z_j > 1$ implies that $x_j > b_{t,j}$,
  meaning that the observation ${\bf x}$ is extreme in the $j$'th dimension.  Note also that
  $\sup_j Z_j$ follows a simple Pareto distribution.

\subsubsection{Limit Measures}
Multivariate EVT assumes the existence of a probability measure $\mu$ on ${\bf Z}$ such that
  \begin{equation}
    \lim_{n\rightarrow\infty}n\text{Pr}\left[\frac{1}{n}{\bf Z} \geq {\bf z}\right] = \mu\left([{\bf 0}, {\bf z}]^c\right).
    \label{eq:standard}
  \end{equation}
  $\mu$ is thus the asymptotic distribution of ${\bf Z}$ in extreme regions.  It exhibits the
  homogeneity property, such that for any region A, $\mu(rA) = r^{-1}\mu(A)$.  By this property we
  thus factorize ${\bf Z}$ into two components:
  \begin{equation}
    \begin{aligned}
      R &= \inorm{{\bf Z}} \in [1, \infty),\\
      {\bf V} &= \frac{{\bf Z}}{R} \in S_{\infty}^{d-1}.
    \end{aligned}
  \end{equation}
  That is, to say, we factorize ${\bf Z}$ into a radial component $R$, and an angular component,
  ${\bf V}$, which is the projection of ${\bf Z}$ onto the positive orthant of the $d$-dimensional
  unit hypersphere defined by the infinity norm, $S_{\infty}^{d-1}$. The radial component $R$, as we
  have stated, follows a simple pareto distribution, and as a consequence of the homogeneity property,
  is independent of the angular component ${\bf V}$.

As the angular component is now independent of the radial component, we can establish a distribution
  on solely the angular component.  For  $B\subset S_{\infty}^{d-1}$, We define the spectral
  (or angular) measure, $\Omega(B)$, as
  \begin{equation}
    \Omega(B) = \mu[{\bf z}: R({\bf z}) > 1, {\bf V} \in B].
  \end{equation}
  \bruno{This needs t be sharpened. I don't see why $R({\bf z})>1$ is needed. Also
  \[	\mu([0,1/\bm{z}]^c) = \int_{{\mathbb S}_\infty^{d-1}}
  \left(\bigvee_{j} \theta_jz_j\right) d\Omega(\theta)\; ,\]
  which is the reason $\Omega$ is a spectral measure.}
  Then we can think of the spectral measure in terms of the limit measure $\mu$, and:
  \begin{equation}
    \mu[{\bf z}:R(z)>t, {\bf V}\in B] = t^{-1}\Omega(B).
  \end{equation}
  Thus we see a one-to-one correspondance between the limit measure $\mu$ and the
    spectral measure $\phi$, and by factoring out the Pareto distributed radial
    component, we can establish a distribution on the angular component
  \begin{equation}
    \text{Pr}\left({\bf V} \in B \mid r > 1\right) = \frac{\Omega(B)}{\Omega(S_{\infty}^{d-1})},
  \end{equation}
  \bruno{Again, this needs sharpening, I don't get the $r>1$. What you have is that, for any value of $r$.
  \begin{equation}
   \lim_{r\rightarrow\infty} \text{Pr}\left({\bf V} \in B \mid R > r\right) =
              \frac{\Omega(B)}{\Omega(S_{\infty}^{d-1})},
  \end{equation}
  So, the point here is what we do in practice. We start by doing the univariate analysis for each
  component. Take $b_{t,j} = F^{-1}(1 - 1/t)$ as the threshold. So
  \begin{equation}
   \lim_{t\rightarrow\infty} \text{Pr}\left({\bf V} \in B \mid R_t > 1\right) =
            \frac{\Omega(B)}{\Omega(S_{\infty}^{d-1})},
  \end{equation}.
  What this means in practice is that we take high thresholds in each dimension, and then take the standardized vectors for which $R>1$.}
  \makenote{Paraphrasing from Goix et. al., figure if/how to cite}
  conditioned on at least one of the components being extreme in the marginal
  sense.  It is using this property that we will establish our method.

For each $\nu\subset\lbrace{1,\ldots,d}, \nu \neq \emptyset$, we define the
    \emph{truncated cone} $\mathcal{C}_{\nu}$, where
  \begin{equation}
      \mathcal{C}_{\nu} = \left\lbrace {\bf z} \geq 0 : \lVert z\rVert_{\infty}\geq 1, z_j \geq 0 \forall j \in \nu, z_j = 0 \forall j \not\in \nu\right\rbrace.
  \end{equation}
  That is, $\nu$ identifies a set index specifying components of the standardized
  data for which the observation is greater than 0.  The observation is greater
  than 0 for columns within that index, and 0 outside that index.  By
  construction, we're also requiring that the observation is greater than 1 in
  at least one of those dimensions.  By this definition, we observe that each
  $\mathcal{C}_{\nu}$ is distinct and disjoint from any other
  $\mathcal{C}_{\nu}$.  Now, defining $\Omega_{\nu}$ as the projection of
  $C_{\nu}$ onto $S_{\infty}^{d-1}$,
  \begin{equation}
      \Omega_{\nu} = \left\lbrace {\bf v} \in S_{\infty}^{d-1} : x_i > 0 \forall i \in \nu, x_i = 0 \forall i \not\in \nu\right\rbrace,
  \end{equation}
  we can clearly see $\mu(\mathcal{C}_\nu) = \Phi(\Omega_{\nu})$ for all
  $\alpha \subset\lbrace1,\ldots,d\rbrace$.  Where we call $\mu(\dot)$ the limit
  measure, we refer to $\Phi(\dot)$ as the \emph{spectral} or
  \emph{angular measure}.  Our goal is establishing statistical inference on this
  angular measure.




% EOF
