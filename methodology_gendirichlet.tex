\subsection{Generalized Dirichlet}
The other natural extension we can form from the Dirichlet model would be to assume data as
  descending from a Generalized Dirichlet distribution.  This model has the same support as the
  Dirichlet, but it is more general in that the rate parameters are not assumed to be the same.
  That is,
  \begin{equation}
    \label{eq:gendirichletrv}
    {\bf x} \sim \text{GD}({\bf x}\mid \zeta, \sigma) =
        \int_0^{\infty}\prod_{l = 1}^d\text{Ga}(rx_l\mid \zeta_l, \sigma_l)\lvert J\rvert\text{d}r
  \end{equation}
Again, the Jacobian is $r^{d-1}$.  As the model is not identifiable otherwise, some restriction
  must be placed on the rate parameters $\sigma$, and the restriction most often used is that the
  first rate parameter $\sigma_1$ is set to 1.

We again consider 2 models under this family: A finite mixture model, and a dirichlet process
  mixture model.

\subsubsection{Finite Mixture of Generalized Dirichlets}
We extend the finite mixture of Dirichlets by, for $l > 1$, allowing $\sigma_{jl}$ to vary, and
  placing Gamma hyperpriors for its generating shape and rate parameters.
  \begin{equation}
    \label{eq:fm_gendirichlet}
    \begin{aligned}
    (r_i, x_i) \mid \delta_i=j &\sim \prod{l = 1}^d\text{Ga}(rx_{il}\mid \zeta_{jl}, \sigma_{jl})\\
        \zeta_{jl} \mid \alpha_l,\beta_l &\sim \text{Ga}(\zeta_{jl}\mid \alpha_l, \beta_l)
        \sigma_{jl} \mid \xi_l, \tau_l &=\sim \text{Ga}(\sigma_{jl}\mid \xi_l, \tau_l)
                                                                  \text{ for }j = 2,\ldots,d
        \alpha_l &\sim \text{Ga}(\alpha_l \mid 0.5, 0.5)\\
        \beta_l &\sim \text{Ga}(\beta_l \mid 2, 2)\\
        \xi_l &\sim \text{Ga}(\xi_l\mid 0.5, 0.5)\\
        \tau_l &\sim \text{Ga}(\tau_l\mid 2, 2)\\
        \lambda &\sim \text{Dir}(0.5)
    \end{aligned}
  \end{equation}
Again, $i$ denotes indexing over observed data, $j$ denotes indexing over clusters, and $l$ denotes
  indexing over dimensions.  Inference conducted on this model is very similar to that of the
  Dirichlet--we augment the data $x_i$ with $r_i$, the latent sum of the independent gammas.
  Doing so allows us to conduct inference on each dimension $l$ independently of the other
  dimensions.  The latent $r$ is generated as
  \begin{equation}
    r_i\mid \zeta, \sigma, \delta_i=j \sim
        \text{Ga}\left(\sum_{l = 1}^d\zeta_{jl},\sum_{l=1}^d\sigma_{jl}x_l\right),
  \end{equation}
This results in a more flexible model as compared to the Dirichlet.  The choice of somewhat
  informative hyperparameters for the rate hyperpriors is to ensure that, for numerical stability's
  sake, rate parameters do not approach 0.

\subsubsection{Dirichlet Process Mixture of Generalized Dirichlets}
This is the final extension to the Dirichlet model, placing a Dirichlet process prior on the
  cluster parameters $\zeta_i,\sigma_i$.  Analogous to the Dirichlet process of Dirichlets, and
  the change made from the finite mixture of Dirichlets to the finite mixture of Generalized
  Dirichlets, this model allows $\sigma_i$ to vary, and places a Gamma hyperprior on its generating
  shape and rate parameters.  We use the same hyperparameters as the finite mixture of generalized
  Dirichlets, assuming that shape parameters descend from a Gamma$(0.5, 0.5)$, and rate parameters
  descend from a Gamma$(2,2)$.






  % EOF
