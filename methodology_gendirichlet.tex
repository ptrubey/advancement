\subsection{Generalized Dirichlet}
The other natural extension we can form from the Dirichlet model would be to assume data as
  descending from a Generalized Dirichlet distribution.  This model has the same support as the
  Dirichlet, but it is more general in that the rate parameters are not assumed to be the same.
  That is,
  \begin{equation}
    \label{eq:gendirichletrv}
    {\bf x} \sim \text{GD}({\bf x}\mid \zeta, \sigma) =
        \int_0^{\infty}\prod_{l = 1}^d\text{Ga}(rx_l\mid \zeta_l, \sigma_l)\lvert J\rvert\text{d}r
  \end{equation}
Again, the Jacobian is $r^{d-1}$.  As the model is not identifiable otherwise, some restriction
  must be placed on the rate parameters $\sigma$, and the restriction most often used is that the
  first rate parameter $\sigma_1$ is set to 1.

We again consider 2 models under this family: A finite mixture model, and a dirichlet process
  mixture model, and we again also consider a vanilla generalized Dirichlet model.

\subsubsection{Finite Mixture of Generalized Dirichlets}
\label{model:mgd}
We extend the finite mixture of Dirichlets by, for $l > 1$, allowing $\sigma_{jl}$ to vary, and
  placing Gamma hyperpriors for its generating shape and rate parameters.  We denote this model as
  \emph{MGD}.
  \begin{equation}
    \label{eq:fm_gendirichlet}
    \begin{aligned}
    (r_i, x_i) \mid \delta_i=j &\sim \prod{l = 1}^d\text{Ga}(rx_{il}\mid \zeta_{jl}, \sigma_{jl})\\
        \zeta_{jl} \mid \alpha_l,\beta_l &\sim \text{Ga}(\zeta_{jl}\mid \alpha_l, \beta_l)
        \sigma_{jl} \mid \xi_l, \tau_l &=\sim \text{Ga}(\sigma_{jl}\mid \xi_l, \tau_l)
                                                                  \text{ for }j = 2,\ldots,d
        \alpha_l &\sim \text{Ga}(\alpha_l \mid 0.5, 0.5)\\
        \beta_l &\sim \text{Ga}(\beta_l \mid 2, 2)\\
        \xi_l &\sim \text{Ga}(\xi_l\mid 0.5, 0.5)\\
        \tau_l &\sim \text{Ga}(\tau_l\mid 2, 2)\\
        \lambda &\sim \text{Dir}(0.5)
    \end{aligned}
  \end{equation}
Again, $i$ denotes indexing over observed data, $j$ denotes indexing over clusters, and $l$ denotes
  indexing over dimensions.  Inference conducted on this model is very similar to that of the
  Dirichlet--we augment the data $x_i$ with $r_i$, the latent sum of the independent gammas.
  Doing so allows us to conduct inference on each dimension $l$ independently of the other
  dimensions.  The latent $r$ is generated as
  \begin{equation}
    r_i\mid \zeta, \sigma, \delta_i=j \sim
        \text{Ga}\left(\sum_{l = 1}^d\zeta_{jl},\sum_{l=1}^d\sigma_{jl}x_l\right),
  \end{equation}
This results in a more flexible model as compared to the Dirichlet.  The choice of somewhat
  informative hyperparameters for the rate hyperpriors is to ensure that, for numerical stability's
  sake, rate parameters do not approach 0.

\subsubsection{Dirichlet Process Mixture of Generalized Dirichlets}
\label{model:dpgd}
Analogous to the DP extension to the finite mixture of Dirichlets, we place a Dirichlet process
  prior on the cluster parameters $\zeta_i,\sigma_i$.  As with \emph{DPD}, and the change made from
  \emph{MD} to \emph{MGD}, this model allows $\sigma_i$ to vary, and places a Gamma hyperprior on
  its generating shape and rate parameters.  We use the same hyperparameters as the finite mixture
  of generalized Dirichlets, assuming that shape parameters descend from a Gamma$(0.5, 0.5)$, and
  rate parameters descend from a Gamma$(2,2)$.

\subsubsection{Generalized Dirichlet with log-normal prior on shape}
We have formed analogous finite mixtures and Dirichlet process mixtures using the Dirichlet kernel,
  along with a log-normal prior on the shape parameters, in the case of the finite mixture, and a
  log-normal centering distribution in the case of the DP mixture.  The rate parameters prior remains
  a product of independent gammas.  That is, for the finite mixture model \emph{MGDLN},
  \begin{equation}
    \begin{aligned}
      (r_i, x_i) \mid \delta_i=j &\sim \prod{l = 1}^d\text{Ga}(rx_{il}\mid \zeta_{jl}, \sigma_{jl})\\
      \zeta_{j} \mid \mu_j, \Sigma_j &\sim \log\mathcal{N}(\zeta_{jl}\mid \mu,\Sigma)
      \sigma_{jl} \mid \xi_l, \tau_l &=\sim \text{Ga}(\sigma_{jl}\mid \xi_l, \tau_l)
                                                                \text{ for }j = 2,\ldots,d
      \xi_l &\sim \text{Ga}(\xi_l\mid 0.5, 0.5)\\
      \tau_l &\sim \text{Ga}(\tau_l\mid 2, 2)\\
      \lambda &\sim \text{Dir}(0.5),
    \end{aligned}
  \end{equation}
  and the Dirichlet Process mixture model, \emph{DPGDLN}
  \begin{equation}
    \begin{aligned}
      (r_i, {\bf x}_i) \mid \zeta_i &\sim \prod_{l = 1}^d\text{Ga}(r_ix_{il}\mid \zeta_{il}, 1)\\
        \zeta_i &\sim \text{DP}(\eta, G)\hspace{2cm}G = \log\mathcal{N}(\zeta_i\mid\mu,\Sigma)
                                          \prod_{l = 2}^d \text{Ga}(\sigma_{il}\mid \xi_l,tau_l)\\
        \alpha_l &\sim \text{Ga}(\alpha_l \mid 0.5, 0.5)\\
        \beta_l &\sim \text{Ga}(\beta_l \mid 2, 2)\\
        \eta &\sim \text{Ga}(\eta \mid 2, \kappa) \hspace{2cm}\kappa \in \lbrace 0.1, 1, 10\rbrace.
    \end{aligned}
  \end{equation}
  






  % EOF
