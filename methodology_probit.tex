
\subsection{Multivariate Normal on Probit-transformed Spherical Coordinate Space}
\label{method:npprobitnorm}
The transformation in Equation~\ref{eqn:spherical} provides us a mapping from $S_{2}^{d-1}$ to a
  $\bm{\theta} \in [0, \pi/2]^{d-1}$.  Constructing a distribution in this geometry--that is, supported
  in this space--offers flexibility.  In Cartesian geometry, we were constrained to distributions supported
  on $\mathcal{R}_+^{d}$, and attempting to integrate out a radial component.  After transforming to
  spherical coordinates, this step is no longer necessary.

The multivariate-normal distribution is often the canonical choice of distributions.  As $[0,\pi/2]^{d-1}$
  is a bound space, to apply the multivariate normal to this space would either require truncating the
  distribution, or further transforming the space.  To this end we apply the probit transformation to
  map $[0,\pi/2]^{d-1}$ to $(-\infty,\infty)^{d-1}$.  For $\theta_i \in [0,\pi/2]$,
  $W_i = \text{Probit}(2\theta / pi)$.  We assume the data as being generated by a mixture of distributions,
  so we place a DP prior on the mean and covariance parameters, with a normal-inverse wishart centering
  distribution.  Conjugate priors are assumed for the parameters of the centering distribution.
  \begin{equation}
    \begin{aligned}
                W_i &\sim \mathcal{N}_{d-1}\left(\mu_i, \Sigma_i\right)\\
    \mu_i, \sigma_i &\sim G_i\\
                G_i &\sim \text{DP}(\eta, G_0(\mu_i,\Sigma_i\mid\mu_0,\Sigma_0))\\
                    &\hspace{1cm}G_0(\mu_i,\Sigma_i\mid\mu_0,\Sigma_0) &=
                      \mathcal{N}_{d-1}(\mu_i\mid\mu_0,\Sigma_0)\text{IW}(\Sigma\mid\nu,\psi)\\
              \mu_0 &\sim \mathcal{N}_{d-1}\left({\bf u},{\bf S}\right)\\
           \Sigma_0 &\sim \text{IW}(\nu_0,\psi_0)\\
               \eta &\sim \text{Ga}(\alpha, \beta)
    \end{aligned}
  \end{equation}
  There is an advantage to this form in that inference on $\mu_i$, $\mu_0$, $\Sigma_i$, and $\Sigma_0$
  are possible with completely conjugate Gibbs steps.  Of course, computational complexity of the
  multivariate normal necessarily limits the scale of data to which this model can be applied.  Fitting
  the model will, among other steps, require matrix inversion, which is an $\mathcal{O}(n^3)$ operation.
  The IVT data sets have 8 and 47 dimensions respectively, so those steps for fitting the model to
  the 47-dimensional data will require approximately 283 times longer than the 8-dimensional data.
  This presents a problem if we wish to scale this model to higher dimensions.

% EOF
