
\subsection{Normal model built on Probit representation of Spherical Coordinate Space}
\label{method:npprobitnorm}
The transformation in Equation~\ref{eqn:spherical} provides us a mapping from $S_{2}^{d-1}$ to a
  $d-1$ dimensional cube, $[0, \pi/2]^{d-1}$.  Building on this transformation allows us to represent
  the data using a true $d-1$ dimensional distribution, rather than generating a latent parameter
  to induce inference in a $d$ dimensional distribution, as we do with all the gamma based models.

A canonical choice of distribution in this space might be to further transform to the $d-1$-hypercube
  to $(-\infty, \infty)^{d-1}$ via marginal probit or logit transformation, and represent the data as
  multivariate normal.  We try that here.  Let $W_i = \text{Probit}(2\theta_i /\pi)$--that is,
  scale $\theta_i$ to the unit interval, then conduct a probit transformation on it to result in
  $W_i \in (-\infty, \infty)$.  Then we establish a multivariate normal distribution on $W$. As we
  are expecting data to descend from a mixture of distributions, we place a DP prior on the multivariate
  normal kernel distribution.  The centering distribution of the DP prior is the product of a
  multivariate normal and inverse Wishart distribution; and we place multivarite normal and inverse
  Wishart priors on these parameters.  As before, we place a gamma prior on the DP concentration
  parameter $\eta$.
  \begin{equation}
    \begin{aligned}
                W_i &\sim \mathcal{N}_{d-1}\left(\mu_i, \Sigma_i\right)\\
    \mu_i, \sigma_i &\sim G_i\\
                G_i &\sim \text{DP}(\eta, G_0(\mu_i,\Sigma_i\mid\mu_0,\Sigma_0))\\
                    &\hspace{1cm}G_0(\mu_i,\Sigma_i\mid\mu_0,\Sigma_0) &=
                      \mathcal{N}_{d-1}(\mu_i\mid\mu_0,\Sigma_0)\text{IW}(\Sigma\mid\nu,\psi)\\
              \mu_0 &\sim \mathcal{N}_{d-1}\left({\bf u},{\bf S}\right)\\
           \Sigma_0 &\sim \text{IW}(\nu_0,\psi_0)\\
               \eta &\sim \text{Ga}(\alpha, \beta)
    \end{aligned}
  \end{equation}
  There is an advantage in that for inference on $\mu_i, \mu_0, \Sigma_i$, and $\Sigma_0$ this model
  is completely conjugate.  However, while the transformation employed in Equation~\ref{eqn:spherical}
  is one to one, small deviations in different dimensions on $S_{\infty}^{d-1}$ will have vastly
  different effects on the resulting transformed variables.  We might think of this as distortion
  induced by the transformation, if we are to evaluate this model in the original space.  This
  distortion may result in an inferior model, when evaluating on $S_{\infty}^{d-1}$.  We will be
  evaluating this model as representative of models on the $d-1$ dimensional coordinate space, and
  comparing it against other models, after projecting its posterior predictive sample back onto
  $S_{\infty}^{d-1}$.

Another practical disadvantage of this model is the need to compute $d-1$-dimensional matrix
  determinants and inversions.  If we consider that inversion is a $\mathcal{O}(n^3)$ operation, and
  finding the determinant is an $\mathcal{O}(n^2)$ operation, we face the real problem of computation
  time climbing astronomically as the number of dimensions grows.  We are currently evaluating this
  model on 8 and 46 dimensions, we can see that those operations on 46 dimensions will take around
  190 times longer than on 8 dimensions.  This presents a problem if we want to fit this model in
  any reasonable time scale.

% EOF
