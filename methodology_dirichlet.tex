\subsection{Dirichlet}
As a distribution defined on the unit hypersphere using the $L_1$ norm, or simplex, Dirichlet is a
  natural choice for our purpose.  The dirichlet random variable can be decomposed as a vector of
  independent gamma random variables with a constant rate parameter, divided by their sum
  (or $L_1$ norm). That is,
  \begin{equation}
    \label{eq:dirichletrv}
    {\bf x} \sim \text{Dir}({\bf x} \mid \zeta) =
        \int_0^{\infty}\prod_{l = 1}^d\text{Ga}(rx_l \mid zeta_l, 1)\lvert J\rvert\text{d}r
  \end{equation}
  The value of the shared rate parameter is irrelevant to the distribution of ${\bf x}$, so by
  convention we set it to one.  The Jacobian for this transformation is $r^{d-1}$.

We consider two mixture models under this family; a finite mixture model of fixed dimension, and an
  infinite mixture model using a Dirichlet process prior for $\zeta$.

\subsubsection{Finite Mixture of Dirichlets}
The finite dirichlet mixture model attempts to represent the distribution of data, projected onto
  the simplex, using a finite mixture of Dirichlet parameters $\zeta$.  That is, for each mixture
  component $j$, we have a vector $\zeta_j$ detailing the parameters of the Dirichlet distribution
  under which observations under that component are distributed.
  \begin{equation}
    \label{eq:fmdirichlet}
    \begin{aligned}
      (r_i, x_i) \mid \delta_i &\sim \prod{l = 1}^d\text{Ga}(rx_{il}\mid \zeta_{jl}, 1)\\
            \zeta_{jl} \mid \alpha_l,\beta_l &\sim \text{Ga}(\zeta_{jl}\mid \alpha_l, \beta_l)
            \alpha_l &\sim \text{Ga}(\alpha_l \mid 0.5, 0.5)\\
            \beta_l &\sim \text{Ga}(\beta_l \mid 2, 2)\\
            \lambda &\sim \text{Dir}(0.5)
    \end{aligned}
  \end{equation}
  Let $i$ denote indexing over the data set.  Let $j$ denote indexing over mixture components.  Let
  Let $l$ denote indexing over columns. Thus, $\delta_i$ denotes the mixture component associated
  with the $i$'th observation.  $\zeta_{jl}$ denotes the shape parameter of the Dirichlet
  distribution associated with mixture component $j$, and column $l$.  The purpose of the rate
  parameter hyperpriors being being somewhat informative, is to ensure for numerical stability
  reasons that the rate parameters do not approach 0.

We perform data augmentation, generating $r_i$, and recovering the original product of independent
  gammas interpretation of the Dirichlet RV.  This enables us to do posterior learning about
  $\zeta_{jl_1}$ independent of $\zeta_{jl_2}$.  The augmented variable, $r_i$, can be generated as
  \begin{equation}
    \label{eq:L1augmentation}
    r_i\mid x, \delta, \zeta \sim \text{Ga}(r_i \mid \sum_{l = 1}^d \zeta_{jl}, 1)
  \end{equation}
  We assign a prior distribution for probability of mixture component membership, $\lambda$, as a
  Dirichlet RV with a relatively weak 0.5 symmetric shape parameter.

The full conditional distribution for $\zeta_{jl}$ is not available in a known form, so sampling
  will require some flavor of MCMC.  We employ a Metropolis-Hastings sampler on $\log\zeta_{jl}$
  using a normal proposal distribution with a standard deviation of 0.3.  This is also employed in
  the posterior sampling for $\alpha_l$. The full conditional for $\beta_l$ arrives in known form
  as a Gamma,
  \begin{equation}
    \label{eq:betafc}
    \beta_l\mid \alpha_l, \zeta \sim \text{Ga}(\beta_l \mid J\alpha + 2, \sum_{j = 1}^J\zeta_{jl} + 2)
  \end{equation}
  The full conditionals for $\zeta_{jl}$ and $\alpha_l$ \makenote{are yet to be inserted, but they
  are simple gamma/gamma models.}

We construct the finite mixture again using data augmentation, introducing the mixture component
  identifier $\delta_i$. The posterior probability that $\delta_i = j$ is constructed as:
  \begin{equation}
    \label{eq:finitemix}
    p(\delta_i = j \mid r, x, \pi, \zeta) \propto \pi_j\prod_{l = 1}^d\text{Ga}(rx_{il}\mid\zeta_{jl},1)
  \end{equation}
  Then the posterior distribution for $\pi$ is formed from the cluster membership identifiers.  Let
  $n_j = \sum 1_{\delta_i = j}$, then $\pi$ is distributed as
  \begin{equation}
    \pi \mid \delta \sim \text{Dir}(n_1 + 0.5, \ldots, n_J + 0.5).
  \end{equation}
  This comprises the simplest model we present for comparison.

\subsubsection{Dirichlet Process Mixture of Dirichlets}
The natural extension to the finite mixture of Dirichlets would be to assume the cluster parameters,
  $\zeta_j$, as descending from an infinite mixture.  As we are simply attempting to represent the
  data, and do not have a compelling interest in controlling the number of clusters, a natural
  choice of prior is the Dirichlet process.  This model will share a great deal of construction,
  posterior inference, and indeed code, with the finite mixture model.
  \begin{equation}
    \label{eq:dpsimplex}
    \begin{aligned}
      (r_i, {\bf x}_i) \mid \zeta_i &\sim \prod_{l = 1}^d\text{Ga}(r_ix_{il}\mid \zeta_{il}, 1)\\
      \zeta_i &\sim \text{DP}(\eta, G)\hspace{2cm}G = \prod_{l = 1}^d \text{Ga}(\zeta_{il}\mid \alpha_l,\beta_l)
      \alpha_l &\sim \text{Ga}(\alpha_l \mid 0.5, 0.5)
      \beta_l &\sim \text{Ga}(\beta_l \mid 2, 2)
      \eta &\sim \text{Ga}(\eta \mid 2, \kappa) \hspace{2cm}\kappa \in \lbrace 0.1, 1, 10\rbrace
    \end{aligned}
  \end{equation}
We denote cluster membership using $\delta_i$, as in the previous case.  Let
  $n_j = \sum 1_{\delta_i = j}$ denote the cluster size.  In the DP literature terminology, we
  are using what is referred to as the collapsed sampler, where for existing clusters,
  \begin{equation}
    p(\delta_i = j \mid r, \zeta, \eta) \propto \frac{n_j}{\sum_j n_j + \eta}
                \prod_{l = 1}^d\text{Ga}(r_ix_{il}\mid\zeta_{jl}).
  \end{equation}
For new clusters, ostensibly we would integrate out the cluster parameters to get a true
  posterior predictive density.  However, doing so in this case is not straitforward.  Instead,
  we employ algorithm 8 from \cite{neal2000}, which introduces another parameter $m$.  We
  generate $m$ new candidate clusters given $\alpha,\beta$, then the probability of $x_i$
  belonging to any particular cluster from the candidate clusters is given as:
  \begin{equation}
    p(\delta_i = j \mid r, \zeta^{\prime}, \eta) \propto \frac{\eta / m}{\sum_j n_j + \eta}
                \prod_{l = 1}^d\text{Ga}(r_ix_{il}\mid\zeta_{jl}^{\prime})
  \end{equation}
  where $\zeta_j^{\prime}$ indicates the cluster parameters from a candidate cluster.  If a new
  cluster is selected, then we append the new cluster parameters to the stack, and continue to the
  next observation.

This model is slightly more complex than the finite mixture of Dirichlets, but at its core it
  employs the same assumption--that the individual columns descend from independent gamma random
  variables, with a fixed rate parameter.

% EOF
