\subsection{Projection onto an arbitrary unit hypersphere}
A hypersphere is a geometric object such that the distance from any point to the center takes a fixed,
  constant value.  The unit hypersphere is a hypersphere where that distance is 1. We can define the
  hypersphere under an arbitrary distance measurement, but for our purpose we will use the class of
  hyperspheres defined under the $\mathcal{L}_p$ norm. Let the $\mathcal{L}_p$-norm be defined as
  \begin{equation*}
    \lVert \bm{s} \rVert_p = \left(\sum_{l = 1}^d \lvert s_l\rvert^p\right)^{\frac{1}{p}}.
  \end{equation*}
  From this, we establish the $\mathcal{L}_1$ norm as $p = 1$, or the absolute sum; the
  $\mathcal{L}_2$ norm, as $p = 2$, the Euclidean distance.  From this we also establish the
  $\mathcal{L}_{\infty}$ norm, as
  \begin{equation*}
    \lVert \bm{s} \rVert_{\infty}
      = \lim\limits_{p\to\infty} \lVert \bm{s} \rVert_p
      = \max_{l\in\lbrace1,\ldots,d\rbrace}s_l.
  \end{equation*}
  We are interested in the direction, or angular component, of vectors described in the positive
  orthant, $\mathcal{R}_{+}^d$.  One means of describing the direction of a vector in $\mathcal{R}_+^d$
  is to project that vector onto $\mathcal{S}_{p}^{d-1}$, the positive orthant of the unit hypersphere
  defined on an $\mathcal{L}_p$-norm, denoted as $\mathcal{S}_{p}^{d-1}$.  That is,
  \begin{equation*}
    \mathcal{S}_{p}^{d-1} = \left\lbrace \bm{s} : \bm{s} \in \mathcal{R}_{+}^{d}, \lVert \bm{s}\rVert_{p} = 1\right\rbrace.
  \end{equation*}
  We project an observation onto this space by dividing said observation by its $p$-norm. Let
  $\bm{x}\in \mathcal{R}_{+}^{d}$, then $\bm{y} = \bm{x} / \lVert \bm{x}\rVert_p \in \mathcal{S}_{p}^{d-1}$.
  We denote the $d-1$ to indicate the loss of one degree of freedom relative to the original vector.
  Observations on one hypersphere can be projected without loss of information onto another by
  dividing by the defining norm of the target hypersphere.

Assuming $\bm{y} \in \mathcal{S}_{p}^{d-1}$, then for finite $p$, $y_d$ can always be represented
  as a function of the other dimensions.  That is,
  \begin{equation*}
    y_d = \left(1 - \sum_{l = 1}^{d-1}y_l^p\right)^{\frac{1}{p}}.
  \end{equation*}
  So the transformation
  \begin{equation}
    \label{eqn:pnormt}
    T(x_1,\ldots,x_d) = \left(\pnorm{\bm{x}}{p}, \frac{x_1}{\pnorm{\bm{x}}{p}},
                          \ldots , \frac{x_{d-1}}{\pnorm{\bm{x}}{p}}\right) = (r,y_1,\ldots,y_{d-1})
  \end{equation}
  does not lose any information.  The reverse of this transformation,
  \begin{equation}
    \label{eqn:pnormtinv}
    T^{-1}\left(r,y_1,\ldots,y_{d-1}\right) =
      \left(ry_1,\ldots,ry_{d-1}, r\left(1 - {\scriptstyle\sum}_{l = 1}^{d-1}y_l^p\right)^{\frac{1}{p}}\right)
  \end{equation}
  equivalently recovers the original data.  To recover the density of the transformed random variables,
  the original density is multiplied by the determinant of the Jacobian--the matrix of derivatives
  of the inverse transformation.  This takes the form:
  \begin{equation}
    \label{eqn:pnormjac}
    r^{d-1}\left[\left(1 - \sum_{l = 1}^{d-1}y_l^p\right)^{\frac{1}{p}} +
        \sum_{l = 1}^{d-1}y_l^p\left(1 - \sum_{l=1}^{d-1} y_l^p\right)^{\frac{1}{p} - 1}\right].
  \end{equation}
  Notice a factor of $r^{d-1}$ independent of $p$. We refer to $\bm{y}$ and $r$ as, respectively,
  the angular and radial components of $\bm{x}$.  If we assume a distribution for $\bm{x}$, then
  by transforming to $r, \bm{y}$ and integrating out $r$, we are left with a distribution on solely
  $\bm{y}$, the projection of $\bm{x}$ onto $\mathcal{S}_{p}^{d-1}$.  If we take the projection to
  its limit as $p\to\infty$, we arrive at a projection onto $\mathcal{S}_{\infty}^{d-1}$.  The models
  we present here follow this form, projecting a distribution in $\mathcal{R}_+^{d}$ to
  $\mathcal{S}_{\infty}^{d-1}$.
%
%   \bruno{\bf The idea of this whole section keeps being unclear. For the Generalized Pareto approach
%   that we are pursuing we need to obtain a distribution on the unit ball for $\infty$ norm. I think
%   that the idea of this section is we can project the original $d$-dimensional vector onto the
%   unit ball for $p$ norm. That induces a distribution on the unit $\infty$ norm ball. You can use the
%   case of the 2-norm as an example.}

% EOF
