\subsection{Projection onto an arbitrary unit hypersphere}
\bruno{\bf How do we get here? I mean, why do we need to talk about hyperspheres? We need a connection with the stuff that comes before this section. }
A hypersphere is a geometric object such that the distance from any point to the center takes a fixed,
  constant value.  The unit hypersphere is a hypersphere where that distance is 1. We can define the
  hypersphere under an arbitrary distance measurement, but let's take the $\mathcal{L}_p$ norm. Let
  the $\mathcal{L}_p$-norm be defined as
  \begin{equation*}
    \lVert {\bf s} \rVert_p = \left(\sum_{l = 1}^d \lvert s_l\rvert^p\right)^{\frac{1}{p}}.
  \end{equation*}
  From this, we establish the $\mathcal{L}_1$ norm as $p = 1$, or the absolute sum, equivalently called
  Manhattan distance; the $\mathcal{L}_2$ norm, as $p = 2$, the Euclidean distance.  From this we
  also establish the $\mathcal{L}_{\infty}$ norm, as
  \begin{equation*}
    \lVert {\bf s} \rVert_{\infty}
      = \lim\limits_{p\to\infty} \lVert {\bf s} \rVert_p
      = \max_{l\in\lbrace{1,\ldots,d}}s_l.
  \end{equation*}
  In the three-dimensional setting, the set of hyperspheres established under $\mathcal{L}_p$,
  $p = 1,2,\ldots,\infty$ will appear as a membrane stretched between $(0,0,1), (0,1,0), (1,0,0)$,
  being blown outward towards $(1,1,1)$ as the pressure, $p$, increases.  At infinite pressure, the
  membrane matches the containing vessel, the cube. \makenote{I don't think this is a good sentence.
  I didn't want to end the paragraph on an equation.  suggestions?}

We are interested in the direction, or angular component, of vectors described in the positive
  orthant, $\mathcal{R}_{+}^d$.  This necessitates the loss of one degree of freedom relative to the
  original vector in $\mathcal{R}_{+}^d$. As we are specifically interested in direction, we can
  project any distribution defined in $\mathcal{R}_{+}^d$ onto the positive orthant of the unit
  hypersphere defined on an $\mathcal{L}_p$-norm, denoted as $\mathcal{S}_{p}^{d-1}$.  That is,
  \begin{equation*}
    \mathcal{S}_{p}^{d-1} = \left\lbrace {\bf y} : {\bf y} \in \mathcal{R}_{+}^{d}, \lVert {\bf y}\rVert_{p} = 1\right\rbrace.
  \end{equation*}
  We can project an observation onto this space by dividing said observation by its $p$-norm. Let
  ${\bf x}\in \mathcal{R}_{+}^{d}$, then ${\bf y} = {\bf x} / \lVert {\bf x}\rVert_p \in \mathcal{S}_{p}^{d-1}$.
  We denote the $d-1$ to indicate the loss of one degree of freedom relative to the original vector.

So $\mathcal{S}_{1}^{d-1}$ defines the unit simplex, $\mathcal{S}_{2}^{d-1}$ defines the generalization
  of a circle--what we would generally refer to as a hypersphere, and $\mathcal{S}_{\infty}^{d-1}$
  the surface of a hypercube. The hyperspheres defined by $\mathcal{L}_p$ as $p$ varies have a one
  to one correspondance with one-another, meaning that observations on one can be projected onto
  another without loss of information.

Assuming ${\bf y} \in \mathcal{S}_{p}^{d-1}$, then for finite $p$, $y_d$ can always be represented
  as a function of the other dimensions.  That is,
  \begin{equation*}
    y_d = \left(1 - \sum_{l = 1}^{d-1}y_l^p\right)^{\frac{1}{p}}.
  \end{equation*}
  So the transformation
  \begin{equation}
    \label{eqn:pnormt}
    T(x_1,\ldots,x_d) = \left(\pnorm{{\bf x}}{p}, \frac{x_1}{\pnorm{{\bf x}}{p}},
                          \ldots , \frac{x_{d-1}}{\pnorm{{\bf x}}{p}}\right) = (r,y_1,\ldots,y_{d-1})
  \end{equation}
  does not lose any information.  The reverse of this transformation,
  \begin{equation}
    \label{eqn:pnormtinv}
    T^{-1}\left(r,y_1,\ldots,y_{d-1}\right) =
      \left(ry_1,\ldots,ry_{d-1}, r\left(1 - {\scriptstyle\sum}_{l = 1}^{d-1}y_l^p\right)^{\frac{1}{p}}\right)
  \end{equation}
  equivalently recovers the original data.  Transforming random variables such that we want to recover
  the new density requires calculating the Jacobian--the matrix of derivatives of the inverse
  transformation--or rather, the determinant thereof.  The determinant of the Jacobian of this
  transformation takes the form
  \begin{equation}
    \label{eqn:pnormjac}
    r^{d-1}\left[\left(1 - \sum_{l = 1}^{d-1}y_l^p\right)^{\frac{1}{p}} +
        \sum_{l = 1}^{d-1}y_l^p\left(1 - \sum_{l=1}^{d-1} y_l^p\right)^{\frac{1}{p} - 1}\right].
  \end{equation}
  Notice a factor of $r^{d-1}$ independent of $p$. We refer to $\bm{y}$ and $r$ as, respectively,
  the angular and radial components of $\bm{x}$.  If we assume a distribution for ${\bf x}$, then
  by transforming to $r, {\bf y}$ and integrating out $r$, we are left with a distribution on solely
  the angular component, or, equivalently, the projection of the vector ${\bf x}$ onto
  $\mathcal{S}_{p}^{d-1}$.
  \bruno{\bf The idea is there, but you need to make the explanation that starts at the beginning of the section and ends here much sharper. }

Many of the models we present here follow this form, where we, for reasons to be elaborated, assume
  a $d$-dimensional Gamma distribution on this hypothetical ${\bf x}$. For finite $p$, this has a
  direct benefit in that it is easy to integrate out $r$.  As we saw with the Jacobian computed
  earlier, no matter what $p$, the Jacobian always has a factor of $r^{d-1}$.  With the independent
  Gamma model, $r$ easily integrates out as a gamma distribution.  We can also perform data
  augmentation generating latent $r$'s, and recovering the ability to do independent inference on
  the parameters of those gamma distributions.  We investigated other unidimensional distributions
  with support on $\mathcal{R}_+$ in the hopes we could perform the same dimension reduction with a
  different parameterization, but none offered the flexibility of the Gamma while allowing $r$ to be
  integrated out in closed form.
  \bruno{\bf You can probably summarize this paragraph with two or three sentences saying that you use a product of independent gamma distributions on the components of $x_j$ for convenience and tractability.}

One might question why we don't use this method to construct a distribution directly on
  $\mathcal{S}_{\infty}^{d-1}$, the unit hypersphere under $\mathcal{L}_{\infty}$.  Put simply, we
  encounter a problem in the transformation.  If we examine the the determinant of the Jacobian
  under the $\mathcal{L}_{p}$ norm, we have a factor along the lines of
  \begin{equation*}
    \left(1 - \sum_{l = 1}^{d-1}y_i^p\right)^{\frac{1}{p} - 1}
  \end{equation*}
  which, if we take the limit as $p$ approaches infinity, if any other $y_l$ than $y_d$ is equal to
  1, then that value approaches $0^{-1}$--an impossibility.  Another way we can recognize the problem
  is directly in the transformation--$T^{-1}(r,y_1,\ldots,y_{d-1})$ will not recover ${\bf x}$ if
  any $y_l$ other than $y_d$ is 1, or equivalently $\max_l x_l\neq x_d$.  Thus, we see a clear
  breaking point between inference conducted on the finite $p$ hypersphere, $\mathcal{S}_{p}^{d-1}$,
  and the $\mathcal{L}_{\infty}$ hypersphere, $\mathcal{S}_{\infty}^{d-1}$.
  \bruno{\bf I am not sure I understand what you are trying to say in the paragraph.}

With this in mind, the way one might build a distribution on $\mathcal{S}_{\infty}^{d-1}$ that still
  operates in Cartesian coordinate geometry might be to include an equal weighting mixture model,
  where each component of the mixture represents the probability of an observation being on that face,
  multiplied by the conditional density of the other dimensions given the face.  That is,
  \begin{equation}
    \label{eqn:mixconddensity}
    f(y) = \sum_{l = 1}^{d}p(y_l = 1)f(y_{-l}\mid y_l = 1)
  \end{equation}
  Under this interpretation, we can consider $p(y_l = 1) = P(x_l = \max_i x_i)$.  Unfortunately,
  this calculation is not straitforward. \bruno{\bf Did you pursue this? I don't think so, so drop it.}

  \makenote{Include arguments from stackexchange thread; cite accordingly}

As an alternative to this projected distribution model, one method we might consider would be to map
  to an alternative geometry, where we can express those $d-1$ degrees of freedom in a $d-1$
  dimensional vector.  In such a space, we can assign whatever model seems appropriate without the
  degrees of freedom restriction necessary in the projected gamma model.  Choosing what $\mathcal{L}_p$
  norm hypersphere we start mapping from identifies what possible geometries we might use.  For instance,
  if we start from the $\mathcal{L}_1$ norm, a possible geometry we might use would be isometric or
  additive logratios\cite{aitchison1982}.  A brief explanation of the additive logratio transformation
  as follows.  For ${\bf x}\in \mathcal{S}_{1}^{d-1}$ (on the unit simplex, so $\sum_{l = 1}^d x_l = 1$),
  let $y_l = \log\frac{x_l}{x_d}$ for $l = 1,\ldots,d-1$. This transformation, and those like it,
  exhibit extreme edge effects.  In this case, if any $x_l\to0$ for $l \neq d$, then $y_l\to-\infty$.
  If $x_d \to 0$, then ${\bf y}\to\infty$.
  \bruno{\bf Did you implement this approach? If not, then drop it as well. You need to focus on the
   approaches that you took. You may add some discussion at the end. Otherwise the narrative becomes confusing.}

\bruno{An example of the approach of using the angular transformation to obtain a distribution on the
hypershere is given by the }i\st{On the} $\mathcal{L}_2$ norm\st{, we consider spherical coordinates}.
  Let ${\bf y} \in \mathcal{S}_{2}^{d-1}$. Then, we transform ${\bf y}$ to spherical coordinates as
  \begin{equation*}
    \theta_l = \arccos\frac{y_l}{\pnorm{{\bf y}_{l:d}}{2}},
  \end{equation*}
  for $l = 1,\ldots, d-1$. That is, the arccosine of the ratio of the mass of a given dimension,
  versus the \emph{combined} mass of that dimension and subsequent dimensions.  If $x_l = 0$, and
  $\pnorm{{\bf x}_{l:d}}{2} > 0$, then $\theta_l = \frac{\pi}{2}$, indicating that all the mass was
  in later dimensions.  Alternatively, if $\theta_l = 0$, then that indicates all the mass was in
  the current dimension.  One can see here that ordering of the axes has a great effect on the
  resulting coordinates.  The inverse of this transformation takes the form:
  \begin{equation}
    \label{eqn:spherical}
    \begin{aligned}
      y_1 &= \cos\theta_1\\
      y_2 &= \sin\theta_1\cos\theta_2\\
      &\vdots\\
      y_{d-1} &= \sin\theta_1\ldots\sin\theta_{d-2}\cos\theta_{d-1}\\
      y_d &= \sin\theta_1\ldots\sin\theta_{d-1}
    \end{aligned}
  \end{equation}
  \bruno{Something is weird with the alignement in this equation. Also, the index of $x$ gets all the way to $d$, but you said $\bm{x} \in S^{d-1}_2$, so $x$ has dimension $d-1$.}
  In this way, we map $\mathcal{S}_2^{d-1}$ to $[0,\pi/2]^{d-1}$.
  \cite{nunez2019} \bruno{use this transformation to obtain a distribution on $[0,2\pi]^{d-1}$}
  \st{follow this course}, starting with \st{the} $d$ independent Gamma distributions and \st{still} integrating out
  $r$ \st{to create an angular distribution constructed on $[0,\pi/2]^{d-1}$}. \bruno{Alternatively, we} can \st{also construct} \bruno{obtain}
  a distribution \bruno{for $\bm{x}$ by considering a multivariate normal distribution for the } \st{directly in this space.  Along this idea, via} probit transformation \bruno{of $\theta_i ,i=1, \ldots , d-1$.} \st{we map
  $[0,\pi/2]^{d-1}$ to $(-\infty, \infty)^{d-1}$, and construct a multivariate normal distribution
  in this geometry.}

% EOF
