\section{Scaling to higher dimensions}
Under the DP mixture of projected Gammas model detailed in Equation~\ref{eqn:infmixp}, computational
  cost of inference using an MCMC sampler as described scales at $n\log n$ with regard to the number
  of observations under consideration, and linearly with regard to the number of dimensions.  This cost
  is exacerbated in the described model by the need to perform inference steps sequentially.  This is
  alleviated somewhat in the finite mixture model in Equation~\ref{eqn:fmixp}, where cost of inference scales
  linearly with regard to both number of observations, and number of dimensions.  Additionally, inference
  steps can be accomplished in parallel.  For these reasons, for a DP model of a given level of performance,
  an equivalently performant finite mixture model will be \emph{faster} to fit.  But even then, for
  computational cost to increase linearly with the number of dimensions, this will to some extent limit
  our ability to conduct inference at large scale.

To motivate the topic of inference at scale, we consider the topic of storm surge.  Innundation, or
  \emph{flooding} can occur as a result of storm activity--winds as a result of the storm pushing water
  on-shore.  This is exacerbated by current tides, and potentially precipitation as a result of the storm.
  With this in mind, designing levees to protect communities and infrastructure will require an
  understanding of what height storm surge can potentially reach.  Simulated data from
  \emph{SLOSH}~\citep{jelesnianski1992} provide information on inundation scenarios, given parameters
  governing the storm as well as local topological features.  An \emph{observation} from a simulation
  run provides the maximum observed level of storm surge at each site, for potentially millions of sites;
  and there are thousands of available simulation runs.  Even subsetting the data to important
  infrastructure will still yield potentially hundreds of thousands of sites \citep{hutchings2021}.

Scaling our models to this level given our current methods of inference is not feasible, both in
  computational complexity given linear scaling, and in model fidelity.   To control computational
  complexity, one option available to us is an alternative inference algorithm such as variational
  Bayes.~\cite{green2015}.  This works by, in relation to our target model, creating an analogous
  variational model that is \emph{easy} to fit, and inference from this model will come as close as
  possible to our chosen model.  This is generally done by minimizing the K.L. divergence between
  the variational distribution and the target distribution.  This presents a difficulty, as it will
  require evaluation of a density on $\mathcal{S}_{\infty}^{d-1}$.

Model fidelity at high dimensions also presents a difficulty--as we observed in
  Figures~\ref{fig:simes,fig:simppl}, the model performance as measured by energy score for the
  Gamma-based models tends to converge as the number of dimensions grows.  We also observed less
  extant clusters on average on the higher-dimensional data, indicating less ability to identify clusters.
  Given this, one method we believe offers some promise for mantaining model fidelity in high dimensions
  would be a more informative prior for the shape parameter.  In Table~\ref{tab:dev}, we saw the
  log-normal prior for the shape parameters dramatically improve model performance on the real data
  in the higher-dimension case.  If we are using an alternative inference method, then the additional
  computational cost from a more complex prior may be alleviated.

% EOF
