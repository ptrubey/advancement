\subsection{Alternative Geometries}
One alternative to projecting a distribution in $\mathcal{R}_+^d$ to $S_{p}^{d-1}$ would be to map
  $\mathcal{S}_{p}^{d-1}$ to an alternative geometry, and fitting a distribution in that space.
  Appropriate alternative geometries depend on $p$--for $p = 1$, one might consider isometric or
  additive logratios~\citep{aitchison1982}.  For $p = 2$, we have spherical coordinates, where we
  map $\bm{x}\in\mathcal{R}_+^d$ to $[0,\pi/2]^{d-1}$.
  \begin{equation*}
      \theta_l = \arccos\frac{x_l}{\pnorm{\bm{x}_{l:d}}{2}},\hspace{1cm}l = 1,\ldots,d-1.
  \end{equation*}
  That is, the arc-cosine of the ratio of mass on a given element, versus the combined mass of that
  and subsequent elements as measured by Euclidean norm.  If $x_l = 0$, and $\pnorm{\bm{x}_{l:d}}{2} > 0$,
  then $\theta_l = \frac{\pi}{2}$, indicating all the mass is on later dimensions.  Alternatively, if
  $\theta_l = 0$, then that means all the mass was on the current element.  One can see here that the
  ordering of axes has a great effect on the resulting coordinates.  The inverse of this transformation
  takes the form:
  \begin{equation}
    \label{eqn:spherical}
    \begin{aligned}
      y_1 &= \cos\theta_1\\
      y_l &= \left[{\textstyle\prod}_{k = 1}^{l-1}\sin\theta_k\right]\cos\theta_l \hspace{1cm}\text{for } l = 2,\ldots,d-1\\
      y_d &= {\textstyle\prod}_{k = 1}^{d-1}\sin\theta_k.
    \end{aligned}
  \end{equation}
  Note that while $\bm{x}\in \mathcal{R}_+^d$, the result of this inverse transformation,
  $y$ exists on $\mathcal{S}_2^{d-1}$. \cite{nunez2019} use this transformation to map the product
  of independent Gammas from $\mathcal{R}^d$ onto $[0,2\pi]^{d-1}$.

% EOF
