\subsection{Alternative Geometries}
An alternative to this projected distribution model, one method we might consider would be to map
  to an alternative geometry, where we can express those $d-1$ degrees of freedom in a $d-1$
  dimensional vector.  In such a space, we can assign whatever model seems appropriate without the
  degrees of freedom restriction necessary in the projected gamma model.  Choosing what $\mathcal{L}_p$
  norm hypersphere we start mapping from identifies what possible geometries we might use.  For instance,
  on the $\mathcal{L}_1$ norm, a possible geometry might be isometric or additive
  logratios~\citep{aitchison1982}.  On the $\mathcal{L}_2$ hypersphere, we consider spherical
  coordinate transformation. Let $\bm{y} \in \mathcal{S}_{2}^{d-1}$. Then, we transform $\bm{y}$
  to spherical coordinates as
  \begin{equation*}
    \theta_l = \arccos\frac{y_l}{\pnorm{\bm{y}_{l:d}}{2}},
  \end{equation*}
  for $l = 1,\ldots, d-1$. That is, the arccosine of the ratio of the mass of a given element,
  versus the \emph{combined} mass of that and subsequent elements.  If $x_l = 0$, and
  $\pnorm{\bm{x}_{l:d}}{2} > 0$, then $\theta_l = \frac{\pi}{2}$, indicating that all the mass was
  in later dimensions.  Alternatively, if $\theta_l = 0$, then that indicates all the mass was in
  the current element.  One can see here that ordering of the axes has a great effect on the
  resulting coordinates.  The inverse of this transformation takes the form:
  \begin{equation}
    \label{eqn:spherical}
    \begin{aligned}
      y_1 &= \cos\theta_1\\
      y_l &= \left[\prod_{k = 1}^{l-1}\sin\theta_k\right]\cos\theta_l \hspace{1cm}\text{for } l = 2,\ldots,d-1\\
      y_d &= \prod_{k = 1}^{d-1}\sin\theta_k
    \end{aligned}
  \end{equation}
  \cite{nunez2019} use this transformation to project a product of Gammas distribution in
  $\mathcal{R}^d$ onto $[0,2\pi]^{d-1}$ Alternatively, we then map these spherical coordiates via
  probit transformation to $(-\infty,\infty)$ to consider a multivariate normal distribution.
  \makenote{Rewrite Paragraph}

% EOF
