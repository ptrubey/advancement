\section{Methodology}
To establish a distribution on this extremal dependence structure, first we must standardize ${\bf X} \in R^d$
  such that its marginal extreme distributions follow a common form.  Following the work of
  \cite{ferreira2014}, if the marginal distributions $F_{l}$ fall into the domain
  of attraction of some GEV $G_l$, then threshold exceedences for a sufficiently high threshold $b_{l}$
  can be modelled as generalized Pareto $H_l$.  Standardization--transforming threshold exceedences to
  standard Pareto--occurs as
  \begin{equation}
    Z_l = \left(1 + \xi_l\frac{X_l - b_{t,l}}{a_{t,l}}\right)_{+}^{1/\xi_l}.
  \end{equation}
  We establish the marginal thresholds $b_{t,l} := \hat{F}_l^{-1}\left(1 - \frac{1}{l}\right)$,
  then fit the scale parameters $a_{t,l}$, and extremal index $\xi_l$ via maximum likelihood.
  Note that $z_l > 1$ implies that $x_l > b_{t,l}$, meaning that the observation ${\bf x}$ is
  extreme in the $l$'th dimension.  Note also that $\sup_j Z_j$ follows a simple Pareto distribution.
  Fitting of the marginal extremal index and scale parameters uses only observations in excess of the
  marginal threshold. Following standardization, further analysis is conditioned on $R \geq 1$.

After standardization and transformation, we are left with $R\sim\text{Pareto}(1)$, and, independently,
  ${\bf V}\in\mathcal{S}_{\infty}^{d-1}$, the positive orthant of the hypersphere under the
  $\mathcal{L}_{\infty}$ norm.  This distribution of ${\bf V}$ effectively describes the dependence
  structure of ${\bf Z}$, so establishing a distribution on ${\bf V}$ becomes the goal of our analysis.
  We are as yet unaware of any known distributions established on this support, so we propose projecting
  a distribution on $\mathcal{R}_{+}^{d}$ onto $\mathcal{S}_{\infty}^{d-1}$, the unit hypersphere
  established under the $\mathcal{L}_{\infty}$ norm.

\subsection{Projection onto an arbitrary unit hypersphere}
A hypersphere is a geometric object such that the distance from any point to the center takes a fixed,
  constant value.  The unit hypersphere is a hypersphere where that distance is 1. We can define the
  hypersphere under an arbitrary distance measurement, but let's take the $\mathcal{L}_p$ norm. Let
  the $\mathcal{L}_p$-norm be defined as
  \begin{equation*}
    \lVert {\bf s} \rVert_p = \left(\sum_{l = 1}^d \lvert s_l\rvert^p\right)^{\frac{1}{p}}.
  \end{equation*}
  From this, we establish the $\mathcal{L}_1$ norm as $p = 1$, or the absolute sum, equivalently called
  Manhattan distance; the $\mathcal{L}_2$ norm, as $p = 2$, the Euclidean distance.  From this we
  also establish the $\mathcal{L}_{\infty}$ norm, as
  \begin{equation*}
    \lVert {\bf s} \rVert_{\infty} = \lim\limits_{p\to\infty} \lVert {\bf s} \rVert_p = \max_{l\in\lbrace{1,\ldots,d}}s_l.
  \end{equation*}
  In the three-dimensional setting, the set of hyperspheres established under $\mathcal{L}_p$,
  $p = 1,2,\ldots,\infty$ will appear as a membrane stretched between $(0,0,1), (0,1,0), (1,0,0)$,
  being blown outward towards $(1,1,1)$ as the pressure, $p$, increases.  At infinite pressure, the
  membrane matches the containing vessel, the cube. \makenote{I don't think this is a good sentence.
  I didn't want to end the paragraph on an equation.  suggestions?}

We are interested in the direction, or angular component, of vectors described in the positive
  orthant, $\mathcal{R}_{+}^d$.  This necessitates the loss of one degree of freedom relative to the
  original vector in $\mathcal{R}_{+}^d$. As we are specifically interested in direction, we can
  project any distribution defined in $\mathcal{R}_{+}^d$ onto the positive orthant of the unit
  hypersphere defined on an $\mathcal{L}_p$-norm, denoted as $\mathcal{S}_{p}^{d-1}$.  That is,
  \begin{equation*}
    \mathcal{S}_{p}^{d-1} = \left\lbrace {\bf y} : {\bf y} \in \mathcal{R}_{+}^{d}, \lVert {\bf y}\rVert_{p} = 1\right\rbrace.
  \end{equation*}
  We can project an observation onto this space by dividing said observation by its $p$-norm. Let
  ${\bf x}\in \mathcal{R}_{+}^{d}$, then ${\bf y} = {\bf x} / \lVert {\bf x}\rVert_p \in \mathcal{S}_{p}^{d-1}$.
  We denote the $d-1$ to indicate the loss of one degree of freedom relative to the original vector.

So $\mathcal{S}_{1}^{d-1}$ defines the unit simplex, $\mathcal{S}_{2}^{d-1}$ defines the generalization
  of a circle--what we would generally refer to as a hypersphere, and $\mathcal{S}_{\infty}^{d-1}$
  the surface of a hypercube. The hyperspheres defined by $\mathcal{L}_p$ as $p$ varies have a one
  to one correspondance with one-another, meaning that observations on one can be projected onto
  another without loss of information.

Assuming ${\bf y} \in \mathcal{S}_{p}^{d-1}$, then for finite $p$, $y_d$ can always be represented
  as a function of the other dimensions.  That is,
  \begin{equation*}
    y_d = \left(1 - \sum_{l = 1}^{d-1}y_l^p\right)^{\frac{1}{p}}.
  \end{equation*}
  So the transformation
  \begin{equation*}
    T(x_1,\ldots,x_d) = \left(\pnorm{{\bf x}}{p}, \frac{x_1}{\pnorm{{\bf x}}{p}},
                          \ldots , \frac{x_{d-1}}{\pnorm{{\bf x}}{p}}\right) = (r,y_1,\ldots,y_{d-1})
  \end{equation*}
  does not lose any information.  The reverse of this transformation,
  \begin{equation*}
    T^{-1}\left(r,y_1,\ldots,y_{d-1}\right) =
      \left(ry_1,\ldots,ry_{d-1}, r\left(1 - {\scriptstyle\sum}_{l = 1}^{d-1}y_l^p\right)^{\frac{1}{p}}\right)
  \end{equation*}
  equivalently recovers the original data.  Transforming random variables such that we want to recover
  the new density requires calculating the Jacobian--the matrix of derivatives of the inverse
  transformation--or rather, the determinant thereof.  The determinant of the Jacobian of this
  transformation takes the form
  \begin{equation*}
    r^{d-1}\left[\left(1 - \sum_{l = 1}^{d-1}y_l^p\right)^{\frac{1}{p}} +
        \sum_{l = 1}^{d-1}y_l^p\left(1 - \sum_{l=1}^{d-1} y_l^p\right)^{\frac{1}{p} - 1}\right].
  \end{equation*}
  \bruno{There is something missing in this formula}\makenote{think it's fixed}
  Notice a factor of $r^{d-1}$ independent of $p$. We refer to $\bm{y}$ and $r$ as, respectively,
  the angular and radial components of $\bm{x}$.  If we assume a distribution for ${\bf x}$, then
  by transforming to $r, {\bf y}$ and integrating out $r$, we are left with a distribution on solely
  the angular component, or, equivalently, the projection of the vector ${\bf x}$ onto
  $\mathcal{S}_{p}^{d-1}$.

Many of the models we present here follow this form, where we, for reasons to be elaborated, assume
  a $d$-dimensional Gamma distribution on this hypothetical ${\bf x}$. For finite $p$, this has a
  direct benefit in that it is easy to integrate out $r$.  As we saw with the Jacobian computed
  earlier, no matter what $p$, the Jacobian always has a factor of $r^{d-1}$.  With the independent
  Gamma model, $r$ easily integrates out as a gamma distribution.  We can also perform data
  augmentation generating latent $r$'s, and recovering the ability to do independent inference on
  the parameters of those gamma distributions.  We investigated other unidimensional distributions
  with support on $\mathcal{R}_+$ in the hopes we could perform the same dimension reduction with a
  different parameterization, but none offered the flexibility of the Gamma while allowing $r$ to be
  integrated out in closed form.

One might question why we don't use this method to construct a distribution directly on
  $\mathcal{S}_{\infty}^{d-1}$, the unit hypersphere under $\mathcal{L}_{\infty}$.  Put simply, we
  encounter a problem in the transformation.  If we examine the the determinant of the Jacobian
  under the $\mathcal{L}_{p}$ norm, we have a factor along the lines of
  \begin{equation*}
    \left(1 - \sum_{l = 1}^{d-1}y_i^p\right)^{\frac{1}{p} - 1}
  \end{equation*}
  which, if we take the limit as $p$ approaches infinity, if any other $y_l$ than $y_d$ is equal to
  1, then that value approaches $0^{-1}$--an impossibility.  Another way we can recognize the problem
  is directly in the transformation--$T^{-1}(r,y_1,\ldots,y_{d-1}) will not recover ${\bf x}$ if
  any $y_l$ other than $y_d$ is 1, or equivalently $\max_l x_l\neq x_d$.  Thus, we see a clear
  breaking point between inference conducted on the finite $p$ hypersphere, $\mathcal{S}_{p}^{d-1}$,
  and the $\mathcal{L}_{\infty}$ hypersphere, $\mathcal{S}_{\infty}^{d-1}$.

With this in mind, the way one might build a distribution on $\mathcal{S}_{\infty}^{d-1}$ that still
  operates in Cartesian coordinate geometry might be to include an equal weighting mixture model,
  where each component of the mixture represents the probability of an observation being on that face,
  multiplied by the conditional density of the other dimensions given the face.  That is,
  \begin{equation*}
    f(y) = \sum_{l = 1}^{d}p(y_l = 1)f(y_{-l}\mid y_l = 1)
  \end{equation*}
  Under this interpretation, we can consider $p(y_l = 1) = P(x_l = \max_i x_i)$.  Unfortunately,
  this calculation is not straitforward.

  \makenote{Include arguments from stackexchange thread; cite accordingly}
  \bruno{You need to sharpen the last two paragraphs because I am having a hard time understanding what you are trying to say here.}
  \makenote{I've tried to make it a little more clear.}

As an alternative to this projected distribution model, one method we might consider would be to map
  to an alternative geometry, where we can express those $d-1$ degrees of freedom in a $d-1$
  dimensional vector.  In such a space, we can assign whatever model seems appropriate without the
  degrees of freedom restriction necessary in the projected gamma model.  Choosing what $\mathcal{L}_p$
  norm hypersphere we start mapping from identifies what possible geometries we might use.  For instance,
  if we start from the $\mathcal{L}_1$ norm, a possible geometry we might use would be isometric or
  additive logratios\cite{aitchison1982}.  A brief explanation of the additive logratio transformation
  as follows.  For ${\bf x}\in \mathcal{S}_{1}^{d-1}$ (on the unit simplex, so $\sum_{l = 1}^d x_l = 1$),
  let $y_l = \log\frac{x_l}{x_d}$ for $l = 1,\ldots,d-1$. This transformation, and those like it,
  exhibit extreme edge effects.  In this case, if any $x_l\to0$ for $l \neq d$, then $y_l\to-\infty$.
  If $x_d \to 0$, then ${\bf y}\to\infty$.

On the $\mathcal{L}_2$ norm, we consider spherical coordinates.  Let ${\bf x} \in \mathcal{S}_{2}^{d-1}$.
  Then, we transform ${\bf x}$ to spherical coordinates as
  \begin{equation*}
    \theta_l = \arccos\frac{x_l}{\pnorm{{\bf x}_{l:d}}{2}},
  \end{equation*}
  for $l = 1,\ldots, d-1$. That is, the arccosine of the ratio of the mass of a given dimension,
  versus the \emph{combined} mass of that dimension and subsequent dimensions.  If $x_l = 0$, and
  $\pnorm{{\bf x}_{l:d}}{2} > 0$, then $\theta_l = \frac{\pi}{2}$, indicating that all the mass was
  in later dimensions.  One can see here that order of the axes has a great effect on the resulting
  coordinates.  The inverse of this transformation takes the form:
  \begin{equation}
    \begin{aligned}
      x_1 &= \cos\theta_1\\
      x_2 &= \sin\theta_1\cos\theta_2\\
      &\vdots\\
      x_{d-1} &= \sin\theta_1\ldots\sin\theta_{d-2}\cos\theta_{d-1}\\
      x_d &= \sin\theta_1\ldots\sin\theta_{d-1}\\
      {\small \prod}_{l = 1}^{d-2}[\sin\theta_{l}]\cos\theta_{d-1}\\
      x_d &= {\small \prod}_{l = 1}^{d-1}\sin\theta_l
    \end{aligned}
  \end{equation}
  In this way, we map $\mathcal{S}_2^{d-1}$ to $[0,\pi/2]^{d-1}$. Nunez \& Antonio\cite{nunez2019}
  follow this course, starting with the independent Gamma distribution and still integrating out
  $r$ to create an angular distribution constructed on $[0,\pi/2]^{d-1}$. But we can also construct
  a distribution directly in this space.  Along this idea, via probit transformation we map
  $[0,\pi/2]^{d-1}$ to $(-\infty, \infty)^{d-1}$, and construct a multivariate normal distribution
  in this geometry.

% EOF


\subsection{Projected Gamma}
\label{method:pg}
Another $d-1$ dimension reduction we can use is instead of projecting onto the unit simplex,
  $S_{1}^{d-1}$, we can project onto the unit hypersphere formed on the Euclidean norm,
  $S_{2}^{d-1}$. \cite{nunez2019} develops this idea fully into the projected gamma distribution.
  Again, we form the distribution as the product of $d$ independent gammas.  That is,
  ${\bf y} = (y_1,\ldots,y_d)^t$, and $y_i\sim\text{Ga}(\alpha_i,\beta_i)$.  We define our
  starting point:
  \begin{equation}
    f({\bf y}\mid{\bf \alpha},{\bf \beta}) = \prod_{j = 1}^d\text{Ga}(y_j\mid\alpha_j,\beta_j),
  \end{equation}
  where $\beta$ is specified as a rate parameter.  \cite{nunez2019} proceeds through a full
  spherical coordinate transformation, where $\theta_i = \cos^{-1}(y_i / \lVert y_{i:d})$,
  for $i\in\lbrace1,\ldots,d-1$.  Then $y_i = r\prod_{j=1}^{i-1}\sin\theta_j\cos\theta_i$.
  This results in a true $d-1$ dimensional distribution, with $\theta_i \in [0, \pi/2)$ for all
  $i\in\lbrace1,\ldots,d-1$.

  $d$-dimensional spherical coordinates ${\bf y} \rightarrow (r,{\bf \theta})$ as
  \begin{equation}
    \label{eqn:transform}
    \begin{aligned}
      y_1     &= r\cos\theta_1,\\
      y_2     &= r\sin\theta_1\cos\theta_2\\
              &\vdots\\
      y_{d-1} &= r\sin\theta_1\ldots\sin\theta_{d-2}\\
      y_{d}   &= r\sin\theta_1\ldots\sin\theta_{d-1}
    \end{aligned}
  \end{equation}
  where $r = \lVert {\bf y}\rVert_{2}$, the euclidean norm of ${\bf y}$.  The inverse of this
  transformation is:
  \begin{equation}
    \label{eqn:invtransform}
    \begin{aligned}
      \theta_1     &= \cos^{-1}\left[\frac{y_1}{\lVert y_{1:d}\rVert_2}\right]\\
      \theta_2     &= \cos^{-1}\left[\frac{y_2}{\lVert y_{2:d}\rVert_2}\right]\\
                   &\vdots\\
      \theta_{d-1} &= \cos^{-1}\left[\frac{y_{d-1}}{\lVert y_{(d-1):d}\rVert_2}\right].
    \end{aligned}
  \end{equation}
  The Jacobian of this transformation is
  \begin{equation*}
    r^{d-1}\prod_{i = 1}^{d-2}(\sin\theta_i)^{d-1-i}.
  \end{equation*}
  This creates the distribution over $r,{\bf \theta}$.  The full conditional for
    $r$ takes the form of a Gamma random variable, and we can integrate it out as
    such.  This leaves the \emph{projected gamma distribution},
  \begin{equation}
    \text{PG}({\bf \theta}\mid{\bf \alpha},{\bf \beta}) = \frac{\Gamma(A)\beta_d^{\alpha_d}}{B^A\Gamma(a_d}\left(\prod_{j = 1}^{d-1}\frac{\beta_j^{\alpha_j}}{\Gamma(\alpha_j)}(\cos\theta_j)^{\alpha_j - 1}(\sin\theta_j)^{(\sum_{h = j + 1}^d\alpha_h) - 1}\right)\mathcal{I}_{(0,\pi/2)^{d-1}}({\bf \theta})
  \end{equation}
  where
  \begin{equation}
    A = \sum_{j = 1}^d\alpha_j \hspace{1cm}\text{and}\hspace{1cm}B = \beta_1\cos\theta_1 + \sum_{j = 2}^{d-1}\left(\beta_j\cos\theta_j\prod_{i = 1}^{j-1}\sin\theta_i\right) + \beta_d\prod_{j = 1}^d-1\sin\theta_j.
  \end{equation}
As is, this model is not identifiable, as taking
  ${\bf \beta}^{(2)} = \alpha {\bf \beta}^{(1)}$ will still yield the same
  distribution of angles. Following \cite{nunez2019}, we have opted to place a
  restriction on $\beta$ such that $\beta_1 := 1$, thus
  ${\bf \beta} = (1, \beta_2, \ldots, \beta_d)^t$.

Inference on this model can take two forms: ${\bf \alpha}$ and ${\bf \beta}$ in
  this form can not be broken down into known-form full conditionals, so we can
  conduct a Metropolis Hastings step for every component, or do a joint proposal
  Metropolis Hastings step for all components at once.  Alternatively, using
  $f(r,{\bf \theta})$, we recognize that $\alpha_i\mid r$ is independent of
  $\alpha_j\mid r$, so we can sample the latent $r$ and conduct independent
  Gibbs steps for each component.  Further, in sampling the $\alpha_j$'s, we can
  integrate out $\beta_j$. Within the Gibbs sampler, we sample $r$, then each
  $\alpha_j\mid r$, then each $\beta_j\mid r, \alpha_j$.  This leads to fast
  convergence, with the only Metropolis Hastings step being for the
  $\alpha_j$'s.  Both $r$ and the $\beta_j$'s are Gamma distributed.

For simplicity, let ${\bf y^{\prime}} = r^{-1}{\bf y}$.  That is,
  ${\bf y^{\prime}}$ is a function of the angular data--from~\eqref{eqn:transform},
  ${\bf y^{\prime}} = {\bf y}/r$, the projection of the ${\bf y}$ vector onto
  the unit hypersphere. We generate a latent $r$, and their product is the
  latent ${\bf y}$.  Given ${\bf y}$, the posterior distributions for
  $(\alpha_i, \beta_i)$, $(\alpha_j,\beta_j)$, $i\neq j$ are independent.

As~\cite{nunez2019} shows, the projected gamma distribution is a flexible model
  for representing data on the positive orthant of the unit hypersphere.  As such,
  given our application restricts us to this domain, one can see that this might be a
  natural choice of distribution for our purpose.

\begin{figure}[h!]
  \centering
  \label{fig:vanillamix}
  \includegraphics[width=5in]{./images/justification_for_more_complex_models}
  \caption{Histograms of Empirical vs Posterior-predictive angular data originating
            from a simulated 3-dimensional gamma dataset.}
\end{figure}

However, as flexible as it is, it alone is not sufficient for our purpose.  Supposing
  a given dataset is the result of two or more generating distributions, then using a
  a single distribution to represent this dataset becomes untenable.  In Figure~\ref{fig:vanillamix}
  we see the empirical distribution a 2-component mixture of projected gammas, plotted
  against the posterior predictive distribution of a projected gamma model fitted to
  that dataset.  As we can see, it has trouble representing the nuances of the two
  component mixture.


% EOF



\subsection{Normal model built on Probit representation of Spherical Coordinate Space}
\label{method:npprobitnorm}
The transformation in Equation~\ref{eqn:transform} provides us a mapping from $S_{2}^{d-1}$ to a
  $d-1$ dimensional cube, $[0, \pi/2]$.  Building on this transformation allows us to represent
  the data using a true $d-1$ dimensional distribution, rather than generating a latent parameter
  to induce a $d$ dimensional distribution, as we do with all the gamma based models.

A canonical choice of distribution in this space might be to further transform to $(-\infty, \infty)$
  via marginal probit or logit transformation, and represent the data as multivariate normal.  We try
  that here.   Let $W_i = \text{Probit}(2\theta_i /\pi)$--that is, scale $\theta_i$ to the unit
  interval, then conduct a probit transformation on it to result in $W_i \in (-\infty, \infty)$.
  Then we establish a multivariate normal distribution on $W$. As we are expecting data to descend
  from a mixture of distributions, we place a DP prior on the multivariate normal kernel
  distribution.  The centering distribution of the DP prior is the product of a multivariate normal
  and inverse Wishart distribution; and we place multivarite normal and inverse Wishart priors on
  these parameters.  As before, we place a gamma prior on the DP concentration parameter $\eta$.
  \begin{equation}
    \begin{aligned}
                W_i &\sim \mathcal{N}_{d-1}\left(\mu_i, \Sigma_i\right)\\
    \mu_i, \sigma_i &\sim G_i\\
                G_i &\sim \text{DP}(\eta, G_0(\mu_i,\Sigma_i\mid\mu_0,\Sigma_0))\\
                    &\hspace{1cm}G_0(\mu_i,\Sigma_i\mid\mu_0,\Sigma_0) &=
                      \mathcal{N}_{d-1}(\mu_i\mid\mu_0,\Sigma_0)\text{IW}(\Sigma\mid\nu,\psi)\\
              \mu_0 &\sim \mathcal{N}_{d-1}\left({\bf u},{\bf S}\right)\\
           \Sigma_0 &\sim \text{IW}(\nu_0,\psi_0)\\
               \eta &\sim \text{Ga}(\alpha, \beta)
    \end{aligned}
  \end{equation}
There is an advantage in that for inference on $\mu_i, \mu_0, \Sigma_i$, and $\Sigma_0$ this model
  is completely conjugate.  However, while the transformation employed in Equation~\ref{eqn:transform}
  is one to one, small deviations in different dimensions on $S_{\infty}^{d-1}$ have vastly different
  effects on the resulting transformed variables.  This induced distortion may result in an inferior
  model, when evaluating on $S_{\infty}^{d-1}$.  We will be evaluating this model as representative
  of models on the $d-1$ dimensional coordinate space, and comparing it against other models, after
  projecting back onto $S_{\infty}^{d-1}$.

Another disadvantage of this model is the need to compute $d-1$-dimensional matrix determinants
  and inversions.  If we consider that inversion is a $\mathcal{O}(n^3)$ operation, we face the real
  problem of computation time climbing astronomically as the number of dimensions grows.  We are
  currently evaluating this model on 8 and 46 dimensions, we can see that those operations on 46
  dimensions will take around 190 times longer than on 8 dimensions.  This presents a problem if we
  want to run this model in any reasonable time scale.

% EOF


% \subsection{Model Comparison on the Hypercube}
It is not immediately obvious which criteria to use to judge these models and decide which best
  represents the data's generating distribution.  We have opted to use the
  \emph{posterior predictive loss} criterion of \cite{gelfand1998} and the
  \emph{energy score} criterion of \cite{gneiting2007}.  Both of these
  metrics require calculating some distance in the target space, and this section will be devoted
  to that end.

\subsubsection{Posterior Predictive Loss Criterion}
The posterior predictive loss criterion, \emph{ppl} is introduced in \cite{gelfand1998}.  When we assume a
  squared error loss function, then for the $l$th observation, the posterior predictive loss criterion
  is computed as
  \begin{equation}
    \label{eq:ppl}
    D_k = \text{Var}(X_l) + \frac{k}{k + 1}\left(\text{E}[X_l] - {\bf x}_l\right)^2,
  \end{equation}
  where $X_l$ is a random variable from the posterior predictive distribution for $x_l$.  The
  scalar $k$ is a weighting factor by which we arbitrarily scale the importance of goodness of fit
  relative to precision.  In our analysis, we take the limit as $k\to\infty$, and thus weight both
  parts equally.  Interpreting this criterion, a smaller $\text{Var}({\bf X}_l)$ indicates a higher
  precision, and a smaller $(\text{E}[{\bf X}_l] - {\bf x}_l)^2$  indicates a better fit.  Thus,
  smaller is better.  Note that this is defined for a univariate distribution--we are going to
  generalize this somewhat, to account both for the multivariate nature of our distribution, and
  its constrained geometry.  That is, let us re-define it as
  \begin{equation}
    \label{eq:ppl2}
    D_k^{\prime} = \text{E}\left\lVert{\bf X_l},\text{E}[{\bf X}_l]\right\rVert_{\Omega}^2 +
                    \frac{k}{k+1}\left\lVert\text{E}[{\bf X}_l],{\bf x}_l\right\rVert-{\Omega}^2.
  \end{equation}
  This generalizes the posterior predictive variance as expected squared distance from a central
  mean.  We can numerically calculate this mean as \makenote{Not sure of this.  googling yields
   this as the result for median...at least in 1 dimension.  not sure how to calculate for mean.}
  \begin{equation*}
    \text{E}[{\bf X}_l] = \text{argmin}_{\omega \in \Omega}\text{E}\lVert {\bf X}_l, s \rVert_{\Omega}.
  \end{equation*}
  This is numerically difficult. Adjusting our interpretation somewhat, we can also project our
  replicates of ${\bf X}_l$ onto the simplex, calculate the mean on the simplex, then re-project
  that point onto the hypercube.

\subsubsection{Energy Score}
The energy score of \cite{gneiting2007} is a generalization of the continuous ranked probability
  score, or \emph{crps}, defined for a multi-dimensional random variable.
  \begin{equation}
    \label{eq:es}
    \text{ES}(P,x) = \frac{1}{2}\text{E}_p\lVert {\bf X}_l,{\bf X}_l^{\prime}\rVert_{\Omega}^{\beta} -
                            \text{E}_p\lVert {\bf X}_l - {\bf x}_l\rVert_{\Omega}^{\beta}
  \end{equation}
  where ${\bf X}_l^{\prime}$ is another replicate from the posterior predictive distribution
  of ${\bf x}_l$. This means, rather than relying on the first and second moments of the posterior
  predictive distribution as in the case of posterior predictive loss, we are instead calculating
  pairwise distances between the observation and draws from the posterior predictive distribution,
  as well as pairwise distances between those replicates themselves.

Now here's the rub.  We are not aware of any standardized distance metrics developed on the
  positive orthant of the unit hypercube.  In the unit simplex, we can assume the use of
  Euclidean norm.  on the unit hypersphere, our task would be slightly more difficult as
  Euclidean norm would under-report the actual distance required for travel between points $a_1$
  and $a_2$.  On the unit hypercube, where our task is defined, the distortion between Euclidean
  norm and the actual distance travelled will be even greater.

The positive orthant of the unit hypercube, defined in Euclidean geometry, is that structure for
  which, in a given point on the hypercube, all dimensions of that point are between 0 and 1, and
  at least one dimension must be 1.  Developing terminology, we can consider observations for which
  the $j$th dimension is equal to 1, to be on the $j$th \emph{face}.  The intersection of the $i$th
  and $j$th face is a hypercube with $d-2$ degrees of freedom, and observations in this space have
  dimensions $i,j$ equal to 1.

A \emph{distance} in this space corresponds to a \emph{geodesic} on this space. From geometry, we
  know that the geodesic, or shortest path between 2 points along the surface of a $d$ dimensional
  figure corresponds to at least one \emph{unfolding}, or \emph{rotation} of the $d$-dimensional
  figure into a $d-1$ dimensional space.  The appropriate term for the structure generated by this
  unfolding is a \emph{net}.  For the appropriate net, a line segment connecting the two points and
  staying within the boundaries of the net corresponds to the shortest path between those points
  \makenote{needs citation!}, and is thus a geodesic.  The length of that line segment is properly
  defines the distance required for travel between those points.

Consider a 3-dimensional cube.  Consider 2 points on this 3-dimensional cube,
  ${\bf a}_1 = (x_1,y_1,z_1)$, and ${\bf a}_2 = (x_2,y_2,z_2)$. Let's say that the two points are
  on the same face.  Then the distance between those two points, the distance one has to travel
  along the space to move from one point to the other, is calculated by Euclidean norm.  Now,
  consider two points on separate faces.  All faces are pairwise adjacent, as we have stated, so
  in order to move to the other point, we must \emph{at least} move to the intersection between the
  faces, then to the other point.  Let ${\bf a}_1$ lie on the $x$ face, and ${\bf a}_2$ lie on the
  $y$ face.  That is, ${\bf a}_1 = (1, y_1, z_1)$, ${\bf a}_2 = (x_2, 1, z_2)$.  Then traveling
  between these points we must at least pass through the intersection of faces $x,y$.  One possible
  net representation of this is unfolding the $y$ face alongside the $x$ face.  We accomplish this
  by applying a rotation and translation to $a_2$, corresponding to the following:
  \begin{equation}
    \label{eq:1drotation}
    a_2^{\prime} = \begin{bmatrix}
    1 \\
    2 \\
    0
    \end{bmatrix}
    +
    \begin{bmatrix}
    0  & 0 & 0 \\
    -1 & 0 & 0 \\
    0  & 0 & 1
    \end{bmatrix}
    \begin{bmatrix}
    x_2 \\
    1 \\
    z_2 \\
    \end{bmatrix} = \begin{bmatrix}
    1 \\
    2 - x_2 \\
    z_2
    \end{bmatrix}
  \end{equation}
  Then, if this is the appropriate net, the distance becomes
  $\lVert {\bf a}_1,{\bf a}_2\rVert = \lVert {\bf a}_1 - {\bf a}_2^{\prime}\rVert_2$ However, there
  is another possible net we must consider, travelling first through the $z$ face then to the $y$
  face.  The rotation for that becomes:
  \begin{equation}
    \label{eq:2drotation}
    a_2^{\prime} = \begin{bmatrix}
    1 \\
    2 \\
    2
    \end{bmatrix}
    +
    \begin{bmatrix}
    0 & 0 & 0  \\
    0 & 0 & -1 \\
    -1 & 0 & 0
    \end{bmatrix}
    \begin{bmatrix}
    x_2 \\
    1 \\
    z_2
    \end{bmatrix} = \begin{bmatrix}
    1 \\
    2 - z_2 \\
    2 - x_2
    \end{bmatrix}
  \end{equation}
Every successive rotation is relative to the last face.  So, as the number of dimensions grows, the
  nmumber of possible rotations grows as well.

  \makenote{need to finish this!}

As we have $d$ dimensions, if 2 observations are on different faces, then there are
  $\sum_{j = 1}^{d-2}\binom(d-2,j) + 1$ possible rotations to consider. \makenote{There are truly $d!$
  possible nets, but when we consider starting and ending faces fixed, and that portions of the net
  that diverge after the ending face are irrelevant, we arrive at that number of rotations that we
  actually need consider}.  This is not insurmountable, it is numerically difficult, and developing
  the generalized rotation strategies for $d$ dimensions is beyond the scope of this analysis.
  However, we are in luck in that all we need for a valid energy score is a
  \emph{negative definite kernel}.  This is defined as a function having symmetry in its arguments,
  $d(x_1,x_2) = d(x_2,x_1)$, and for which $\sum_{i =1}^n\sum_{j=1}^na_ia_jd(x_1,x_2) \leq 0$ for
  all positive integers n, with the restriction that $\sum_{i=1}^na_i = 0$.  The Euclidean norm
  is one example of a negative definite kernel.

Let's go back to the first rotation--we held that as a numerically easier analogue to what we were
  actually calculating--the distance from the starting point, to some optimal point along the
  intersection between the $x$ and $y$ planes, to the ending point.  That is, the sum of two
  Euclidean norms.



% EOF



\input{methodology_applications}

\subsection{Spatial Threshold Modeling}
Following the work of \cite{ferreira2014}, any of the above methods can be
  extended to the spatial domain by modification of the marginalization process.
  Assume a spatial process $X({\bf s})$, ${\bf s}\in {\bf S}$.  Then take the
  transformation
\begin{equation}
  \label{eqn:spatial}
  Z({\bf s}) = \left(1 + \gamma({\bf s})\frac{X({\bf s}) - b_t({\bf s})}{a_t({\bf s})}\right)_{+}^{1/{\gamma({\bf s})}}
\end{equation}
  where $b_t({\bf s})$ is a function corresponding to a high threshold at location
  ${\bf s}$, analogous to the role $b_t$ played previously. $a_t({\bf s})$
  corresponds to a scaling function, and $\gamma({\bf s})$ an extremal value index
  function.

% EOF
