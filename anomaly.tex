\section{Applications to Anomaly Detection}
\label{sec:anomaly}
Anomaly detection describes a field of methods for identifying observations as \emph{anomalous}.  That
  is to say, observations which do not fit the general distribution of the data from which they came.
  Equivalent names for this field are \emph{outlier detection}, and \emph{novelty detection}, though
  some authors will separate the use of those terms by whether or not it is assumed that anomalous
  observations are included in the training dataset.  Crucial to this distinction is the assumption
  of a \emph{clean} training data set.

The applicability of extreme value theory to anomaly detection is predicated on the belief that
  extreme observations are more susecptible to anomalies.  \cite{goix2017} provides a discussion to
  this effect, that extreme observations exist at the border between anomalous and non-anomalous regions.

For our purpose, we do not assume the existence of labels in the training dataset, and seek an
  algorithm that can produce anomaly scores in an absence of class labels. As such, we will offer
  a brief overview of unsupervised anomaly detection methods, as well as discussion of the methods
  we are preparing here as competing models.

\subsection{Existing Methods}
The complete field of anomaly detection is wide, and we could not hope to provide a comprehensive review
  it.  However, most methods can be roughly grouped into three main ideas: statistical models, clustering
  methods, and non-statistical models. Core to the development of a method lies a basic question,
  \emph{what is an anomaly?}  The answer to this question depends on application and what we hope to
  learn.  For instance, in a regression setting, observations that produce large outliers may be considered
  anomalous.  In a classification setting, common to most methods of anomaly detection is the belief
  that data belonging to regions of relative data sparsity are more likely to be anomalous.  Methods in
  this space work to find these regions of relative data sparsity, or equivalently, regions of relative
  data abundance.

\subsubsection{Statistical Models}
Statistical modelling for anomaly detection employs attempts to model the distribution of
  data, with the goal that the model can be to estimate the data density around an observation.
  The assumption made is that observations in regions of low estimated density are more likely to be
  anomalous.  This can include non-parametric density estimation routines such as the $k$-Nearest
  Neighbors, or \emph{$k$-NN}~\citep{kramer2013}; kernel density estimation such as the Parzen-Rosenblatt
  windowing method~\citep{parzen1962,rosenblatt1956}.  This can also include parametric density estimation
  routines, such as Gaussian mixture models~\citep{mcnicholas2010}.

\subsubsection{Clustering Methods}
Clustering methods tend to sort data into groups \emph{near} each-other.  These rely on \emph{distance}
  metrics and tend to make no assumptions regarding the underlying distribution of the data.  We can
  further sub-divide this sub-field into density-based, centroid-based, and linkage-based clustering
  methods.  For all of these methods, we can regard observations with less association to their identified
  cluster as being more anomalous.

Linkage-based clustering methods group data based on pairwise distance point-to-point, or between
  elements of clusters.  \cite{ackerman2010} offers a review of the topic.  An illustrative example
  is single linkage, where the distance between two clusters is defined as the minimum distance between
  a point in each set.   Similarly, complete linkage defines the metric to be the maximum pairwise
  between a point in each set.  The goal of the linkage-based clustering algorithm is to maximize the
  total distance between clusters under whatever metric of distance is used, along with minimizing
  distance within clusters.  An observation's anomaly score might be a function of distance to its nearest
  neighbor within its assigned cluster, or distance to its assigned cluster's centroid.

Centroid based clustering methods instead generate $k$ cluster centroids according to some metric,
  then an observation's \emph{anomaly} score is a function of the distance from that
  observation to the nearest cluster centroid.  The algorithm used to find the cluster
  centroids is also variable, changing what metric is optimized in finding the location of the
  centroids.  The very popular \emph{$k$-Means} \citep{hartigan1979} is an example of this type of
  method.  Under $k$-Means, cluster assignment is decided by minimizing within-cluster distance,
  which for a given $k$ simultaneously maximizes between-cluster distance.

Density based methods will use pairwise distances to establish some measure of local density, then
  establish local modes as clusters.  \emph{DBSCAN} \citep{ester1996} follows this, by forming
  neighborhoods of observations and establishing labels based on the neighborhood.

\subsubsection{Non-Statistical Models}
There are additional non-statistical families of methods beyond clustering, adapted from the realm
  of general classification models to unsupervised learning.  The Isolation Forest,\citep{liu2000}
  adapted from random forests~\citep{breiman2001} uses decision trees to isolate observations--the
  observations that are more easily isolable are regarded as more anomalous.  One-class SVM
  \citep{chang2011} is a variant of the support vector machine classification system optimized towards
  anomaly detection. In a similar manner to the support vector machine using support vectors to
  describe a decision boundary separating classes of data, the one-class SVM uses support vectors to
  describe a decision boundary around \emph{normal} behavior.  A higher distance to that decision
  boundary on the anomalous side is regarded as \emph{more anomalous}.

\subsection{Proposed Scoring Methods}
We offer two adaptations of scoring methods for use on the hypersphere.  As we have previously
  mentioned, an evaluable density is difficult to create on $\mathcal{S}_{\infty}^{d-1}$.
  \bruno{This paragraph is incomplete}

\subsubsection{Density Estimation on the hypersphere}
One way we can think of anomalies is as observations that are unlikely to occur.  Given a fitted model,
  those observations that come from a low density area may be more likely to be \emph{anomalous}.
  As we do not have a density available in closed form, we turn to a non-parametric kernel density
  estimation method for establishing local density, based on KNN.  We generate the inverse scores as
  local posterior predictive density (as established by the KNN estimator), times the probability of
  observing \bruno{observing what?}

We establish the local posterior predictive density under the KNN estimator by assuming a locally
  uniform density within a $d-1$ dimensional ball, centered on $V_i$, with a radius defined by
  the hypercube distance estimation metric detailed in~\ref{subsec:distance}.
  \begin{equation}
    \label{eqn:ad_knn}
      S_{i,(k)}^{-1} =
        \frac{k}{N}\frac{\Gamma\left(\frac{d-1}{2} - 1\right)}{\pi^{\frac{d-1}{2}}D_{k}^{d-1}(V_i)}
  \end{equation}
  This provides an estimator of local density that we can use to establish a scoring metric.

\subsubsection{Contribution to posterior predictive loss}
Another way we can think of anomalies is as observations that are difficult to fit--that is, difficult
  for a fitted model to predict.  We assume that the greater the difficulty encountered in fitting a
  given observation, the more anomalous it is.  The posterior predictive loss criterion in
  Equation~\ref{eq:ppl} provides one such metric.  The first term measures the posterior predictive
  variance--a larger variance corresponds to less precision in the prediction, or alternatively, a lower
  density in the predictive distribution.  The second term measures squared bias--the squared distance
  between the average predicted value, and the observation.  Both of these terms provide a metric of
  how \emph{difficult} a particular observation is to fit.

\subsection{Anomaly Detection Performance on the hypersphere}
As a preliminary analysis, we have prepared simulated AD datasets using Algorithm~\ref{algo:simulated}.
  for 95 percent of the data, we apply the algorithm directly; then for the remaining 5 percent of the
  data we mutate the generating distribution by rotating axis labels by 1 for each mixture component.
  This results in \emph{anomalies} that are comparatively rare, and come from a different distribution
  than \emph{normal} data.

\begin{table}[h]
  \centering
  \label{tab:ad_sim_results}
  
\begin{tabular}{ccccccc}
\toprule
\multicolumn{2}{c}{ } & \multicolumn{2}{c}{Proposed} & \multicolumn{3}{c}{Classic} \\
\cmidrule(l{3pt}r{3pt}){3-4} \cmidrule(l{3pt}r{3pt}){5-7}
nMix & nCol & kNN & PPL & o.c.SVM & L.O.F. & I.Forest\\
\midrule
5 & 5 & {\bf 0.750} & 0.734 & 0.511 & 0.596 & 0.508\\
5 & 10 & {\bf 0.754} & 0.711 & 0.554 & 0.740 & 0.524\\
10 & 5 & 0.532 & 0.543 & 0.574 & {\bf 0.599} & 0.563\\
10 & 10 & {\bf 0.605} & 0.593 & 0.509 & 0.544 & 0.459\\
\bottomrule
\end{tabular}

  \caption{Area under the ROC Curve for various methods, established on simulated data.}
\end{table}

The higher the number of mixture components, the more comparatively \emph{noisy} the data.  In such
  an environment, anomalies are more difficult to spot.  We see a relative performance drop moving
  from 5 to 10 mixture components, across all scoring algorithms.





% EOF
