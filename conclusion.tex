
\section{Conclusion}

\makenote{Snippets}


A brief explanation of the additive logratio transformation
as follows.  For $\bm{x}\in \mathcal{S}_{1}^{d-1}$ (on the unit simplex, so $\sum_{l = 1}^d x_l = 1$),
let $y_l = \log\frac{x_l}{x_d}$ for $l = 1,\ldots,d-1$. This transformation, and those like it,
exhibit extreme edge effects.  In this case, if any $x_l\to0$ for $l \neq d$, then $y_l\to-\infty$.
If $x_d \to 0$, then $\bm{y}\to\infty$.


\subsection{Spatial Threshold Modeling}
Following the work of \cite{ferreira2014}, any of the above methods can be
  extended to the spatial domain by modification of the marginalization process.
  Assume a spatial process $X({\bf s})$, ${\bf s}\in {\bf S}$.  Then take the
  transformation
\begin{equation}
  \label{eqn:spatial}
  Z({\bf s}) = \left(1 + \gamma({\bf s})\frac{X({\bf s}) - b_t({\bf s})}{a_t({\bf s})}\right)_{+}^{1/{\gamma({\bf s})}}
\end{equation}
  where $b_t({\bf s})$ is a function corresponding to a high threshold at location
  ${\bf s}$, analogous to the role $b_t$ played previously. $a_t({\bf s})$
  corresponds to a scaling function, and $\gamma({\bf s})$ an extremal value index
  function.


% EOF
