
\section{Conclusion}
In this paper, we have built upon the definition of the multivariate Pareto described in \cite{ferreira2014}
  to establish a useful parametric representation of its dependence structure.  That dependence structure
  is supported on the positive orthant of the unit hypersphere under the $\mathcal{L}_{\infty}$ norm,
  for which we demonstrated the inherent difficulty in establishing a distribution supported thereon.
  To build that representation, we presented a distribution based on a product of independent Gammas,
  projected onto the same support as the dependence structure, and established a negative definite kernel
  appropriate to the space to facilitate a means of model selection under the energy score criterion.

The opportunities afforded us by a simple parametric representation of the dependence structure included
  a tractable representation of pairwise dependence, as well as conditional survival curves--the
  inversion of which produces the more well known \emph{return rates}.

We leveraged the information afforded us by that distribution of the dependence structure to build
  an efficient anomaly detection algorithm, that in preliminary analysis we found competitive with
  existing methods.

We finally postulated a path towards large scale inference using our projected Gamma distribution.
  This path presents its own difficulties in the need for an evaluable density.  For future work, we
  believe model fidelity for high-dimensional data can be improved with appropriate priors.
  
\bruno{You need to write this less as a set of conclusions and more as a proposal of work for which you
have obtained preliminary results. So, change the tone. Also, you need a timeline of how you are
going to proceed from now.}

\makenote{Make this longer.}




% \makenote{Snippets}
%
%
% A brief explanation of the additive logratio transformation
% as follows.  For $\bm{x}\in \mathcal{S}_{1}^{d-1}$ (on the unit simplex, so $\sum_{l = 1}^d x_l = 1$),
% let $y_l = \log\frac{x_l}{x_d}$ for $l = 1,\ldots,d-1$. This transformation, and those like it,
% exhibit extreme edge effects.  In this case, if any $x_l\to0$ for $l \neq d$, then $y_l\to-\infty$.
% If $x_d \to 0$, then $\bm{y}\to\infty$.
%
%
% \subsection{Spatial Threshold Modeling}
% Following the work of \cite{ferreira2014}, any of the above methods can be
%   extended to the spatial domain by modification of the marginalization process.
%   Assume a spatial process $X({\bf s})$, ${\bf s}\in {\bf S}$.  Then take the
%   transformation
% \begin{equation}
%   \label{eqn:spatial}
%   Z({\bf s}) = \left(1 + \gamma({\bf s})\frac{X({\bf s}) - b_t({\bf s})}{a_t({\bf s})}\right)_{+}^{1/{\gamma({\bf s})}}
% \end{equation}
%   where $b_t({\bf s})$ is a function corresponding to a high threshold at location
%   ${\bf s}$, analogous to the role $b_t$ played previously. $a_t({\bf s})$
%   corresponds to a scaling function, and $\gamma({\bf s})$ an extremal value index
%   function.
%

% EOF
