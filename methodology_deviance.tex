\subsection{Proper Scoring Rules on the Hypercube}
It is not immediately obvious which criteria to use to judge these models and decide which best
  represents the data.  We have opted to use the \emph{posterior predictive loss} criterion of
  \cite{gelfand1998} and the \emph{energy score} criterion of \cite{gneiting2007}.  Both of these
  metrics require calculating some approximation of distance in the target space, and this
  section will be devoted to that.

We are not aware of any standardized distance metrics developed on the positive orthant of the
  unit hypercube.  In the unit simplex, we can assume the use of Euclidean norm.  on the unit
  hypersphere, our task would be slightly more difficult as Euclidean norm would under-report
  the actual distance required for travel between points a and b.  On the hypercube, assuming
  points $a$ and $b$ lie on different faces, the distortion between Euclidean norm and the
  actual distance required for travel will be even greater.

The positive orthant of the unit hypercube, defined in Euclidean geometry, is that structure for
  which, in a given point on the hypercube, all dimensions of that point are between 0 and 1, and
  at least one dimension must be 1.  Developing terminology, we can consider observations for which
  the $j$th dimension is equal to 1, to be on the $j$th \emph{face}.  The intersection of the $i$th
  and $j$th face is the $d-2$ dimensional hypercube for which observations in this space have
  dimensions $i,j$ equal to 1.

The \emph{shortest path} in this space will be the projection of the straight line defined 



% EOF
